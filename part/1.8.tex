\section{ Уровень третий}

Как показывает практика, до третьего уровня добираются далеко не все. Задания этого уровня достаточно сложны и требуют от школьников уже не только владения материалом, но и значительной доли фантазии и сообразительности. 
\head{Cентябрь}{Листок 1. Четность. Уровень 3.}

В этом разделе собраны задачи, для решения которых придется применить несколько идей, разобранных в листках уровней 1 и 2.
\begin{thm}
	Лёша нарисовал на клетчатой бумаге замкнутый путь, идущий по линиям сетки. Докажите, что он нарисовал четное число единичных отрезков (единица -- сторона клетки).
\end{thm}
\begin{thm}\label{1.23}
	Улитка ползет по плоскости с постоянной скоростью. Каждые 15 минут она поворачивает под прямым углом. Докажите, что вернуться в исходную точку она сможет только через целое число часов.
\end{thm}
\begin{thm}
	Кузнечик прыгает а) по прямой на  метр вправо или влево; б) по плоскости, каждым прыжком перемещаясь на метр на север, юг, запад или восток; в)  по узлам клетчатой плоскости, каждым прыжком перемещаясь по диагонали одной из клеток. Может ли он вернуться в исходную точку через 2011 прыжков?
\end{thm}
\begin{thm} а) б) в)
	Может ли в условиях предыдущей задачи кузнечик вернуться в исходную точку через 2010 прыжков?
\end{thm}
\begin{thm}
	${}^{n}$ Дорожки парка -- линии квадратной сетки. Одна ячейка -- 100 на 100 метров. Войти в парк можно через единственный вход, а выйти -- через единственный выход. Петя и Вася делились впечатлениями по поводу прогулок и выяснили, что один прошел по дорожкам на 300 м меньше, чем другой, причем каждый обошел целое число дорожек (то есть, ступив на дорожку, проходил до конца без возвращений). Не ошиблись ли ребята?
\end{thm}
\begin{thm}\label{1.27}
	К 17-значному числу прибавили число, записанное теми же цифрами, но в обратном порядке. Докажите, что хотя бы одна из цифр полученной суммы четна.
\end{thm}
\begin{thm}
	17 девочек и 17 мальчиков встали в хоровод. Докажите, что у кого-то с обеих сторон стоят девочки.
\end{thm}
\begin{thm}
	Маша и ее друзья встали в круг. Оказалось, что у каждого из них оба соседа либо оба мальчики, либо девочки. Мальчиков среди Машиных друзей пять. А сколько девочек?
\end{thm}
\begin{thm}
	Во время перемирия за круглым столом разместились рыцари двух враждующих станов. Оказалось, что число рыцарей, справа от которых сидит враг, равно числу рыцарей, справа от которых сидит друг. Докажите, что число рыцарей делится на 4.
\end{thm}
\begin{thm}\label{1.31}
	В классе 30 учеников. Они сидят за 15-ю партами. При этом оказалось, что ровно половина всех девочек сидит с мальчиками. Докажите, что их не удастся пересадить (за те же 15 парт) так, чтобы ровно половина всех мальчиков класса сидела с девочками.
\end{thm}
\begin{thm}\label{1.32}
	Сто грустных мартышек кидают друг в друга одним кокосовым орехом. Грустная мартышка, попавшая орехом в другую грустную мартышку, становится веселой и больше не грустнеет. Мартышка, в которую попали, выбывает из игры. Каких мартышек больше выбыло из игры -- веселых или грустных -- к моменту, когда в игре осталась одна мартышка?
\end{thm}
\newpage