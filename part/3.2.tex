\head{Ноябрь}{Листок 1. Комбинаторика.}

\textit{В этот раз задачи листочка каждого уровня  рассчитаны на одно занятие. Другими словами, сдавшие 1 уровень и решающие второй, имеют времени только одно занятие - сегодняшнее. Сдающие также и первый имеют в свoем распоряжении два занятия - это и еще следующее. После чего все задачи обоих уровней будут разобраны.}

\begin{thm}
	Сколько диагоналей и сторон у выпуклого  а) 23-угольника;~~~ б) n-угольника?
\end{thm}

\begin{thm}
	Сколько есть возможностей поставить на шахматную доску\\ а) белую и черную~~~б) две черных ладьи так, чтобы они не били друг друга?~~~в) Укажите отличия между пунктами а) и б).
\end{thm}

\begin{thm}
	Сколько существует шестизначных чисел с разными цифрами?
\end{thm}

\begin{thm}
	У нас есть 13 шариков - 12 черных и 1 зеленый. Найдите число а) всех перестановок; б) всех различных перестановок.\footnote{Одинаковыми считаются те, при которых шары с одинаковым номерами одного цвета.}
\end{thm}

\begin{thm}
	Та же задача, что и предыдущая, но не один зеленый шарик, а два (всего получается 11 черных и 2 зеленых).
\end{thm}

\begin{thm}
	Шася знает 14 дебютов, 11 миттельшпилей и 13 эндшпилей. Сколь разнообразна Шасина игра в шахматы?\footnote{Дебют - начало партии, миттельшпиль - середина, эндшпиль - окончание.}
\end{thm}

\head{Ноябрь}{Листок 1. Комбинаторика. Проверочная работа.}


\begin{thm}	
	Будем считать число неинтересным, если в его записи присутствуют только четные цифры. Сколько всего неинтересных пятизначных чисел?
\end{thm}

\begin{thm}	
	В алфавите племени Мумба-Бумба всего три буквы: Б, У и М. Словом считается любая не пустая последовательность не более чем из четырех букв. Сколько всего слов в языке племени Мумба-Бумба?
\end{thm}

\begin{thm}
	22 лампочки расположены в ряд. Сколькими способами мы можем зажечь две из них?
\end{thm}

\begin{thm}	
	Сколькими способами можно выбрать в племени 7Ю класса из 20 человек а) вождя и его советника; б) двух дозорных?
\end{thm}

\begin{thm}	
	В Доминандии, как и у нас, есть игра в домино, только точек на домино не от 0 до 6, а от 0 до 2011. Сколько всего доминошек в этой игре?
\end{thm}

\begin{thm}	
	Сколько в мире десятизначных чисел, у которых все цифры разные?
\end{thm}

\begin{thm}	
	Чему равно а) $17!\times 18$;  б) $8!\times 90$;  в) $(N - 2)!\times (N - 1)$?
\end{thm}

\begin{thm}\label{3.4}
	Вычислите а) $\frac{2011!}{2010!}$ ;  б) $\frac{(N+1)!}{(N-1)!}$ ; в)  $\frac{(N+1)!-N!}{N}$
\end{thm}

%\begin{prf}
%	в) $\frac{(N+1)!-N!}{N}=\frac{N!\times (N+1)-N!}{N}=\frac{((N+1)-1)N!}{N}=\frac{N\times N!}{N}=N! $
%\end{prf}

\begin{thm}	\label{3.5}
	Может ли число $N!$ оканчиваться а) одной пятеркой;~~ б) одной восьмеркой;~~ в) семью нулями?
\end{thm}

%\begin{prf}
%	а) Если число оканчивается пятеркой, то по признаку делимости на 5 оно должно делиться на 5. Следовательно, если какой-то факториал оканчивается на 5, то среди его множителей должна быть пятерка. Поэтому это не может быть менее, чем $ 5! $. Но $ 5! $ Оканчивается нулем и, соответственно, последняя цифра равна нулю во всех последующих факториалах.\\  б) Из пункта а) следует, что все факториалы больше 5! Не могут быть искомыми, так как не оканчиваются на 8. Поэтом достаточно проверить оставшиеся первые 5 факториалов: $ 0! = 1 $, не подходит; $ 1! = 1 $ - не подходит; $ 2! = 2 $ - не подходит; $ 3! = 6 $ - не подходит; $ 4! = 24 $ - не подходит. Тем самым доказано, что такого быть не может. в) Заметим, что ноль на конце дает произведение 5 и 2. Поскольку в разложении любого факториала пятерок меньше, чем двоек, то количество нулей на конце совпадает с количеством пятерок в разложении. Посмотрим, какие числа могут давать эти пятерки: 5; 10; 15; 20 - по одной пятерке; 25 - две пятерки. Поэтому факториалы от 0! до 24! Имеют на конце не более 4 нолей, а от 25! до 29! - ровно 6. Факториалы от 30! до 34! Имеют ровно 7 нолей, что требовалось выяснить.
%\end{prf}
\hrule
\section{Решение некоторых задач проверочной}

\textbf{Задача \ref{3.4}}
	Вычислите а) $\frac{2011!}{2010!}$ ;  б) $\frac{(N+1)!}{(N-1)!}$ ; в)  $\frac{(N+1)!-N!}{N}$

\begin{prf}
	в) $\frac{(N+1)!-N!}{N}=\frac{N!\times (N+1)-N!}{N}=\frac{((N+1)-1)N!}{N}=\frac{N\times N!}{N}=N! $
\end{prf}

\textbf{Задача \ref{3.5}}
	Может ли число $N!$ оканчиваться а) одной пятеркой;~~ б) одной восьмеркой;~~ в) семью нулями?

\begin{prf}
	а) Если число оканчивается пятеркой, то по признаку делимости на 5 оно должно делиться на 5. Следовательно, если какой-то факториал оканчивается на 5, то среди его множителей должна быть пятерка. Поэтому это не может быть менее, чем $ 5! $. Но $ 5! $ Оканчивается нулем и, соответственно, последняя цифра равна нулю во всех последующих факториалах.\\  б) Из пункта а) следует, что все факториалы больше 5! Не могут быть искомыми, так как не оканчиваются на 8. Поэтом достаточно проверить оставшиеся первые 5 факториалов: $ 0! = 1 $, не подходит; $ 1! = 1 $ - не подходит; $ 2! = 2 $ - не подходит; $ 3! = 6 $ - не подходит; $ 4! = 24 $ - не подходит. Тем самым доказано, что такого быть не может. в) Заметим, что ноль на конце дает произведение 5 и 2. Поскольку в разложении любого факториала пятерок меньше, чем двоек, то количество нулей на конце совпадает с количеством пятерок в разложении. Посмотрим, какие числа могут давать эти пятерки: 5; 10; 15; 20 - по одной пятерке; 25 - две пятерки. Поэтому факториалы от 0! до 24! Имеют на конце не более 4 нолей, а от 25! до 29! - ровно 6. Факториалы от 30! до 34! Имеют ровно 7 нолей, что требовалось выяснить.
\end{prf}