\section{ Письменные задачи.}
Если вы заметили, в листке есть задача, отмеченная буквой ${}^{n}$ -- это означает, что решение задачи должно быть выполнено в письменном виде и сдано на листке.

Для чего это сделано?

Общеизвестно, что большинству школьников достаточно сложно дается запись своих решений. И такие задания направлены как раз на вырабатывание этого умения. В том числе этим регулируется строгость изложения. Школьникам сообщается, что в письменных решениях будет учитываться не только «решил» / «не решил», а насколько связен и логичен текст. 

В любом случае потом всем школьника выдаются письменные решения «от автора», чтобы они могли сравнить со своими опусами.


\section{ Переход на второй уровень}

Наконец наступает момент, когда кто-то решил все задачи листка уровня 1 или просто решил, что он уже все знает из этого листка и готов к проверке.

Наступает момент проверочной работы.

\head{Сентябрь}{Листок 1.1 Четность. Переводная работа.}

\begin{enumerate}
	\item Дайте определение чётного числа.
	
	\item Какие из утверждений верны: А) Сумма нечётного количества чётных чисел нечётна. Б) Произведение нечётного количества чётных чисел нечётно. В) Сумма нечётного количества нечётных чисел нечётна. Г) Произведение чётного количества нечётных чисел чётно.
	
	\item Сумма двух чисел нечетна. Какой четности может быть их разность?
	
	\item Не выполняя никаких арифметических действий,~укажите чётность числа: 
	\[(124789254651-45645646)\cdot(67776-59681)+(18963-712)\cdot(94978-8569)+8865431\cdot735+17\] 
	\item Среди 11 целых чисел 8 четных и 3 нечетных. Какой максимальной длины цепочка последовательных чисел может быть выстроена из них в лучшем случае? 
	\item  Аня попала в Зазеркалье, где встретила свое отражение -- Яну. Потом Яна попала в свое Зазеркалье, где встретила свое отражение -- конечно же, Аню-2! Аня-2 попала в свое Зазеркалье, где была Яна-2. И так происходило достаточно долго, пока зеркало не разбилось. Назовите, как звали 2019-ю девочку?
\end{enumerate}
\hrulefill

\section{Комментарии для преподавателя} 

Первая проверочная работа достаточно простая. Она рассчитана на 25-35 минут. 

Полные пояснения требуются только в задачах 5 и 6. В остальных только ответы.

Если все задания выполнены правильно, то считается, что школьник успешно прошел 1 уровень и получает листок 2 уровня.

Если же что-то неверно, то те, кто не сдавали задачи из листка 1 уровня,  должны их решить и сдать, а те, кто сдал, но тем не менее работу написал неудачно, получает Дубль листка 1 уровня. Поскольку у нас пока не было прецедентов, что кто-то не справился с проверочной работой, то мы не обзавелись дублем. Однако учитель легко может его сделать, заменив имена и числа в задачах.


\section{ Решения переводной работы с уровня 1 на уровень 2}
\begin{enumerate}
	\item Четное число дает остаток 0 при делении на 2.
	\item A) нет, потому что сумма любого количества четных чисел -- четна.~~ Б) нет, потому что произведение целых чисел четно, если хотя бы одно из них четно.~~ B) да ~~Г) нет, потому что произведение любого количества нечетных чисел -- нечетно.
	\item Только нечетной, потому что их четности различны.
	\item Перепишем в виде: $\overbrace{\underbrace{(Н-Ч)}_{Н}\cdot\underbrace{(Ч-Н)}_{Н}}^{Н}+\overbrace{\underbrace{(Н-Ч)}_{Н}\cdot\underbrace{(Ч-Н)}_{Н}}^{Н}+\underbrace{(Н\cdot Н)}_{Н}+Н=Н+Н+Н+Н=Ч$
	\item Четные и нечетные чередуются, значит четных может быть только на 1 больше или меньше нечетных, а нечетных максимум 3. Значит четных 4. Значит всего последовательных чисел не больше 7.
	\newpage
	\item Выпишем первые несколько имен:
	\begin{center}
		1 девочка -- Аня-1\\
	2 девочка -- Яна-1\\
	3 девочка -- Аня-2\\
	4 девочка -- Яна-2\\
	5 девочка -- Аня-3\\
	6 девочка -- Яна-3\\
	...\\
	\end{center}
Заметим, что Аня всегда "нечетная девочка", а Яна - всегда "четная". После этого можно заметить, что номера девочек - это номер соответствующего четного или нечетного числа в последовательности. Например, Яна-5 обозначает пятое четное число, а Аня-7 -- седьмое нечетное.
	Соответственно задача сводится к тому, чтобы узнать каким по счету является данное число.
	2010 -- 1005 четных и 1005 нечетных числе, следовательно 2011 -- 1006 нечетное число, значит ему соответствует девочка Аня-1006
	2018 -- 1009 четных и 1009 нечетных числе, следовательно 2019 -- 1010 нечетное число, значит ему соответствует девочка Аня-1010
	
	\textit{Примечание.} Если не получается решить задачу в общем случае, можно построить примеры при малых числах и найти в них закономерность. Предложить гипотезу и доказать ее.
\end{enumerate}