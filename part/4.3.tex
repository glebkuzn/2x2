\head{Ноябрь}{Листок 4. Теория чисел. Уровень 3+.}

\begin{thm}Докажите, что
    \par
    а)~число тогда и только тогда делится на 4, когда две его последние цифры образуют число, \\ делящееся на 4.
    \par
    б)~число тогда и только тогда делится на 8, когда число, составленное из трех последних его цифр делится на 8.
\end{thm}

\begin{thm}
    Попробуйте сформулировать и доказать признаки делимости на 7 и на 13.
    \\ 
    {\footnotesize\textit{(\underline{Подсказка}: $1001=7\times11\times13$)}}
\end{thm}

\begin{thm}
    Перед боем с белогвардейцами у Василия Ивановича и Петьки было поровну патронов. Василий Иванович израсходовал в бою в 8 раз меньше патронов, чем Петька, а осталось у него в 9 раз больше патронов, чем у Петьки. Докажите, что изначально количество патронов у Василия Ивановича делилось на 71. 
\end{thm}

\begin{thm}
    Автоматический хозрасчётный калькулятор предоставляет следующие услуги:
    \par 
    I)~Умножение имеющегося числа на три за 5 коп.; 
    \par
    II)~Прибавление к имеющемуся числу четырёх за 2 коп. 
    \par
    Число 1 вводится бесплатно. Какую наименьшую сумму нужно потратить, чтобы получить число 2012?
\end{thm}

\begin{thm}$^{\ast}$
Назовём автобусный билет с шестизначным номером счастливым, если сумма цифр его номера делится на 7. Могут ли два билета подряд быть счастливыми?
\end{thm}

\begin{thm}$^{\ast}$
Шайка разбойников отобрала у купца мешок с монетами. Каждая монета стоит целое число грошей. Оказалось, что какую монету не отложи, оставшиеся монеты можно поделить между разбойниками так, что каждый получит одинаковую сумму. Докажите, что число монет без одной делится на число разбойников в шайке.
\end{thm}

\begin{thm}
    У царя Дадона в одиночных камерах сидели 100 пленников. Поворот ручки отпирает каждую камеру, следующий поворот запирает, ещё один снова отпирает и т.д. К празднику царь решил освободить часть пленников и накануне послал слугу, который повернул ручку на двери каждой камеры. Все двери оказались отперты. Но тут пришел второй посыльный и повернул ручку каждой второй камеры. Двери камер 2, 4, 6, ... вновь оказались заперты. Следующий посланец повернул ручки камер 3, 6, 9, 12 и т.д. Ещё один - в каждой четвёртой камере. То же повторяли следующие посланцы вплоть до сотого, повернувшего ручку сотой камеры. Наконец наступил праздник, и сидевшие в открытых камерах вышли на свободу. Сколько пленников освободил Дадон?
\end{thm}

\begin{thm}
    Верны ли следующие части «признака делимости на 27»:
    \par
    а)~если сумма цифр числа делится на 27, то число делится на 27;
    \par
    б)~если число делится на 27, то сумма его цифр делится на 27
\end{thm}

\begin{thm}
    Мальчик Костя живёт в 20-этажном доме. После того, как Костя однажды покатался в лифте, в нём стали работать только две кнопки: «+5» (при нажатии на эту кнопку лифт поднимается на 5 этажей вверх, если это возможно), и «-7» (при нажатии на неё лифт опускается на 7 этажей вниз). Можно ли, пользуясь таким лифтом, попасть
    \par
    а)~с первого этажа на второй? 
    \par
    б)~со второго этажа на первый? 
    \par
    в)~А можно ли вообще пользоваться этим лифтом, т.е. позволяет ли он добираться с любого этажа на любой другой?\footnote{Карикатура из газеты «Большой Новосибирск».}
\end{thm}

\begin{thm}
    $a, b, c$ - нечётные натуральные числа, не являющиеся квадратами.
    \par
    Может ли число $a^b b^c c^a$ быть полным квадратом?
\end{thm}

\begin{thm} $^{\ast}$
    Верно ли, что существует число, кратное 2012, и имеющее сумму цифр, равную 2012?
\end{thm}

\begin{thm}
    Число состоит из 36 цифр. Разрешается разбить его на группы из шести цифр и как-нибудь переставить эти группы. Известно, что одна из перестановок в семь раз больше другой. Докажите, что эта большая перестановка делится на 49.
\end{thm}

\begin{thm} $^{\ast}$
    У каждого марсианина три руки и несколько антенн. Несколько марсиан взялись за руки так, что каждый взял за руки трёх других, и все руки при этом оказались заняты. При этом выяснилось, что количество антенн у любых двух взявшихся за руки марсиан отличается ровно в 6 раз. Может ли суммарное количество антенн у марсиан быть ровно 2006?
\end{thm}

\textbf{\textit{Для получения оценки «5» нужно решить 8 задач,  
для получения еще одной «5» - все 13}.}