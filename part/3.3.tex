\head{Ноябрь}{Листок 2. Комбинаторика.}
\begin{thm}
	Пусть теперь дано 9 черных и 4 зеленых шарика. Выясните, сколько существует перестановок, при которых первые 3 шарика зеленые?
\end{thm}

\begin{thm}	
	а) Cколько всего существует различных перестановок 13 шариков из предыдущей задачи?~~~ б) А сколькими способами можно выбрать 3 шарика из 13-ти, разложенных в ряд, шариков?~~~в) Укажите отличия между пунктами а) и б).
\end{thm}

\begin{thm}
	Давайте, заменим 10 черных шариков из задачи 3.10 на 10 разноцветных (получится 13 шариков, среди которых 2 зеленых и 11 других цветов). Вопрос: сколько теперь различных перестановок?
\end{thm}

\begin{thm}
	Теперь добавим 3 синих шарика к 13-ти из задачи 3.14 (будет 16 шариков 12-ти цветов). Тот же вопрос: сколько теперь различных перестановок?
\end{thm}

\begin{thm}	
	Возьмем 10 шариков, среди которых 3 зеленых, 2 черных, 2 белых и 3 других различных цветов. Как много разных перестановок в этом случае?
\end{thm}

\begin{thm}	
	Сколько различных слов (не обязательно осмысленных) можно получить, переставляя буквы в слове МАТЕМАТИКА?
\end{thm}

\begin{thm}	
	Есть на свете некто, кого зовут Словак Разнобукевич. Если взять его фамилию, и стереть несколько букв\footnote{Как минимум одну, но не все!}, сколько различных фамилий может получиться?
\end{thm}

\begin{thm}\label{3.17}
	На бумажной полоске написано слово АБВГДЕЙКА. Сколькими способами можно разрезать эту полоску на три слова? \footnote{Не обязательно осмысленных}
\end{thm}

%\begin{prf}
%	Резать полоску можно только между буквами, поэтому по сути дела задача сводится к выбору промежутков между букв, где нужно поставить перегородки. Так как требуется разрезать на три слова, то надо выбрать два промежутка. Букв всего девять, следовательно, промежутков восемь. Таким образом, задача свелась к задаче о том, сколькими способами можно выбрать два предмета из восьми. Первый предмет можно выбрать восьмью способами, а второй уже только семью. Всего $8\times7=56$ способов. Но каждый из способов мы сосчитали дважды (например, выбор предмета №6 и №8 можно осуществить так: выбрать сначала №6, а потом №8, или наоборот - сначала №8, а затем №6). Поэтому мы должны полученный результат разделить на два. Итак, окончательный \textit{\underline{ответ:}} 28 способов.
%\end{prf}

\begin{thm}
	Дана полоска $АБВ \dots ЭЮЯ$. (всего 33 буквы). Сколько способов разрезать эту полоску на 3 слова?\footnote{Смысл слов не важен.}
\end{thm}

\section{Аналогии}

\begin{thm}
	$^n$Сколькими способами можно разложить 17 одинаковых шаров в четыре ящика? (в каждый ящик нужно положить хотя бы один шар)
\end{thm}

\begin{thm}
	Сколькими способами можно представить число 17 в виде 4 натуральных слагаемых, если способы, отличающиеся порядком слагаемых, различны?\footnote{Например: $4+4+4+5$ и $4+5+4+4$ - разные случаи.}
\end{thm}

\begin{thm}
	Дана шахматная доска $33\times33$, причем левая нижняя клетка черная. Тогда такой вопрос: сколько есть вариантов положить 3 черные шашки на белые поля в самой верней горизонтали доски?
\end{thm}

\begin{thm}
	Гадкий Миша решил испортить всем новый год - он хочет разорвать елочную гирлянду, состоящую из 22 лампочек, на 4 части (в каждой части должна быть хотя бы одна лампочка, иначе Мише будет неинтересно!) А мы хотим узнать, сколькими способами с такой затеей он может испортить праздник?
\end{thm}

\begin{thm}
	Маленькая Шура решила устроить праздник своим родителям. Для этого она выгребла все монеты из свинки-копилки, пошла в магазин, где продавались воздушные шарики, но обнаружила, что ей не хватает денег, чтоб купить весь магазин, а хватает ровно на 4 шарика. И вот проблема: всего шариков 22 и все разных цветов…. Итак, сколько вариантов у Шуры выбрать 4 шарика из 22 шариков на прилавке?
\end{thm}

\begin{ex}
	Найдите все аналогии из вышеприведенных задач.
\end{ex}

\head{Ноябрь}{Листок 2. Комбинаторика. Проверочная работа}
\begin{enumerate}
	\item В 7Ю учится 20 человек. \\
	a.	Сколькими способами их можно выстроить в ряд? \\
	b.	Сколькими способами их можно выстроить в ряд, если Лиза и Катя хотят стоять рядом? \\
	c.	если Лиза хочет стоять правее Кати (не обязательно рядом)?\\
	d.	если Ваня хочет стоять правее Глеба, а Глеб правее Миши?
\item	В 7Ю учится 20 человек. Сколькими способами можно разбить их на две  равные команды\\
	a.	если эти команды называются Фениксы и Сфинксы.\\
	b.	если мы не различаем эти команды?\\
	c.	Если в Сфинксах и Фениксах может быть не равное число участников.
\item	В 7Ю учится 20 человек - 10 мальчиков и 10 девочек.\\
	a.	Сколькими способами они могут разбиться на пары?\\
	b.	если в паре должны быть мальчик и девочка?\\
	c.	В классе 12 парт. Сколькими способами они могут рассесться в классе?\\
	d.	если они должны сидеть по двое?\\
	e.	если они должны сидеть по двое - мальчик с девочкой?
\item	В 7Ю учится 20 человек. Иван Андреевич хочет выставить им четвертные оценки.\\
	a.	Сколькими способами он это может сделать, если он ставит 2, 3, 4 и 5?\\
	b.	если он не хочет ставить 2?\\
	c.	Если он хочет поставить не больше трех двоек?
\end{enumerate}
	 