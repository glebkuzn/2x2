\head{Сентябрь}{Теоретический Листок. Четность.}

Прежде чем решать задачи вспомните, какие числа называются четными, а какие нечетными и их простейшие свойства. 


\begin{table}[h]\centering
	\begin{tabular}{|c|}
		\hline
		\textit{Сумма или разность двух чисел одной четности четна.}\\
		\textit{Сумма или разность двух чисел разной четности нечетна.}\\
		\hline
	\end{tabular}
\end{table}

Решите несколько приведенных ниже упражнений:

\begin{ex}
	Сложили 3 нечетных числа. Могло ли получится 1024?
\end{ex}

\begin{ex} Не выполняя никаких арифметических действий, назовите чётность чисел: 
	\begin{enumerate}
		\item $1000-947\times7567\times6+2009+2006$
		\item $204\times121+5360\times7+3121+6731\times81\times11-154-77+87$ 
		\item $(1246254651-45645645)\times(67876-59681)+(1163-712)\times(948-8569)+886541\times735+1$
		\item$1000-947\times7567\times76+2009+2006$
		\item$204\times2121+5360\times7+3121+6731\times81\times11-154-77+87$
		\item$(1246254651-45645645)\times(67876-59681)+(1163-712)\times(948-8569)+886541\times735+1$
		
	\end{enumerate}
\end{ex}

\begin{ex}
	Имеется два числа одной четности. Какова может быть четность их разности?
\end{ex}

\begin{ex}
	Сумма двух чисел нечетна. Какой четности их разность?
\end{ex}

\begin{ex}
	Двое играют в следующую игру: первый игрок рисует на клетчатой бумаге квадрат. Затем второй игрок зачеркивает одну из клеток этого квадрата. Потом то же делает первый, и так далее. Проигрывает тот, кто не сможет сделать ход. Кто выиграет при правильной игре и как ему надо играть?
\end{ex}

\begin{thm}
	Парламент состоит из двух равных по численности палат. На совместном заседании присутствовали все, и никто не воздержался при голосовании. Когда было объявлено, что некоторое решение было принято большинством в 23 голоса\footnote{«большинством в 23 голоса» – значит голосующих «за» было на 23 больше, чем голосующих «против».},  оппозиция закричала «Это обман!». Почему?
\end{thm}

\begin{prf}
	На первый взгляд, не обладая информацией о численности палат, вряд ли можно делать какие-либо выводы. Тем не менее, предположим, что описанная в задаче ситуация возможна. Если голосовавших «против» было х, то голосовавших «за» было $x + 23$. Следовательно, всего проголосовало $2x + 23$ человек (нечетное число), а в двух палатах равной численности в сумме четное число членов. Противоречие.
\end{prf}
\begin{prf}
\textit{\textbf{«без формул или на пальцах»}} Первоначальное утверждение: если какое-то количество можно разбить на две равные части, то это количество четно. В нашем случае из условия следует, что в парламенте четное число членов. Далее, заметим, что если к какому-либо числу прибавить нечетное число, то четность суммы поменяется (если число было четным – станет нечетным, если же было нечетным, то станет четным). Тогда количество проголосовавших «за» и «против» имеют разную четность. Но тогда их сумма нечетна, как сумма двух чисел разной четности. Противоречие.
\end{prf}
\vfill
В приведенном примере мы использовали наблюдение, что сумма или разность четного и нечетного числа – нечетное число. Если рассматривать четность как частный случай делимости (а именно: четность – это делимость на натуральное число 2), то в данном наблюдении нет ничего необычного. Отличие четности от любой другой делимости заключается в наблюдении, что сумма двух нечетных чисел – четное число (например, сумма двух чисел, не делящихся на три, может как делиться, так и не делиться на 3). Продемонстрируем еще на одном примере, как можно это использовать.

\begin{thm}
	На 99 карточках пишут числа 1, 2, ..., 99, перемешивают их, раскладывают чистыми сторонами вверх и снова пишут числа 1, 2, ..., 99. Для каждой карточки складывают два её числа и 99 полученных сумм перемножают. Докажите, что результат чётен.
\end{thm}

\begin{prf}
	Так как на карточках написаны числа от 1 до 99, то среди них нечетных на одно больше чем четных. Поскольку нечетных больше половины, на какой-то карточке с обеих сторон будут нечетные числа. Их сумма будет четным числом. При умножении нескольких натуральных чисел, среди которых есть четное, получается четное число. 
\end{prf}

В задаче 2 мы использовали также тот факт, что произведение нескольких натуральных чисел, среди которых есть четное, – четное число. Это же можно формулировать по-другому: если произведение некоторого набора натуральных чисел – число нечетное, то все эти числа нечетные.


\begin{table}[h]\centering
	\begin{tabular}{|c|}
		\hline
		\textit{Сумма нечетного количества нечетных чисел нечетна.}\\
		\hline
	\end{tabular}
\end{table}

Этот факт является простым следствием того, что сумма двух нечетных чисел – четное число.

\begin{ex}
	Сложили 5 целых чисел. Получили 2017. Сколько среди них может быть нечётных? А если чисел 2017?
\end{ex}

\begin{ex}
	\label{u7}
	Сколько нечетных среди первых 100 натуральных	чисел?
	А среди 2019?  
	А среди первых N натуральных чисел?
\end{ex}

\begin{thm}
	Филя перемножил 17 целых чисел и получил 1025, а Степашка сложил эти же числа и получил 100. Докажите, что кто-то из них ошибся.
\end{thm}

\begin{prf}
	Предположим, что Филя не ошибся. Тогда если Филя, перемножая натуральные числа, получил нечетный результат, то все множители были нечетными. Но в то же время сумма 17 нечетных чисел – нечетное число, и никак не может равняться 100.
\end{prf}

\begin{center}
	{\large\textbf{Чередование}}
\end{center}

Выполняя \textit{упражнение \ref{u7}}, вы, несомненно, воспользовались тем, что четные и нечетные числа в числовом ряду идут попеременно.  
%\begin{table}[h]\centering
%	\begin{tabular}{|c|c|c|c|c|c|c|}
%		\hline
%		1&2&3&4&5&6&...\\
%		\hline
%		Н&Ч&Н&Ч&Н&Ч&...\\
%		\hline
%	\end{tabular}
%\end{table}

\begin{ex}
	Укажите, в каких еще задачах, приведенных ранее, используется идея чередования. 
\end{ex}

\begin{ex}
	Среди 10 целых чисел 7 четных и 3 нечетных. Какой максимальной длины цепочка последовательных чисел может быть выстроена из них в лучшем случае?
\end{ex}

\begin{ex}
	На шахматной доске на одной из клеток стоял конь. Он сделал несколько ходов и вернулся в ту же клетку. Четное или нечетное число ходов он сделал?
\end{ex}

\begin{ex}
	Дрессированный кузнечик прыгает по прямой – каждый раз на 1 метр вправо или влево. Через некоторое время он оказался в исходной точке. Докажите, что он сделал четное число прыжков.
\end{ex}

\begin{thm}
	За время летних каникул сторож посадил вдоль школьного забора 20 яблонь. 1 сентября оказалось, что число яблок на соседних деревьях отличается на 1. Может ли на всех этих яблонях быть ровно 2017 яблок?
\end{thm}

\begin{prf}
	Не может. Будем называть яблоню с четным количеством яблок четной, а с нечетным – нечетной. Предположим, что описываемое в условии возможно. Тогда четные и нечетные яблони чередуются. Следовательно, половина яблонь – четные, а половина – нечетные. Половина – это 10. Сосчитаем общее количество яблок. Нам нужно сложить 10 четных чисел и 10 нечетных. Очевидно, что эта сумма является четным числом, поэтому получить 2017 невозможно.    
\end{prf}

\textbf{\textit{Ответы к упражнениям:}}
\textbf{1.} нет. 
\textbf{2.} 1) нечетно; 2) четно; 3) нечетно. ~
\textbf{3.} любым четным числом. ~
\textbf{4.} нечетна. ~
\textbf{5.} Первый всегда может обеспечить себе выигрыш. Для этого он должен нарисовать квадрат, состоящий из четного числа клеток. ~
\textbf{6.} 1) 1 или 3, или 5. 2) любое нечетное число от 1 до 2017. ~
\textbf{7.} 1) 50; 2) 1009; 3) если N четно, то N/2, если же нечетно, то (N+1)/2. ~
\textbf{8.} например, задача 2 и упр.5. ~
\textbf{9.} цепочка из 7 чисел. ~
\textbf{10.} четное. ~