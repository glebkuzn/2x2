\head{Сентябрь}{Листок 1. Четность. Уровень 1.}
\epigraph{\textit{Если написанная программа сработала правильно, то это значит, что во время ее работы выполнилось четное число ошибок или программист не понял задание.}}{Правило четности ошибок}

\textit{Задачи вы можете решать как дома, так и в классе. Рекомендуется коротко записывать решение в тетрадь. Рассказывать решения задач вы будет одному из принимающих. Задачи, отмеченные значком п, приниматься в устном форме не будут. В данном случае это единственная задача \ref{1.4}. Решение таких задач надлежит выполнить в письменном виде дома и принести на занятие спецматематики. Через некоторое время часть задач будет разобрана, после чего решения этих задач приниматься не будет. Рекомендуем решать задачи по порядку, поскольку зачастую в решении следующей задачи можно использовать идею из предыдущей.}
\begin{flushright}
	\textit{Желаем успеха!}
\end{flushright}

\begin{thm}
	Филя пишет на доску одно целое число, а Степашка - другое. Если произведение чётно, победителем объявляют Филю, если нечётно, то Степашку. Может ли один из игроков играть так, чтобы непременно выиграть?
\end{thm}

\begin{thm}
	Докажите, что произведение любых двух последовательных чисел четно.
\end{thm}

\begin{thm}
	Каким (четным или нечетным) может быть число  $n^2 + n$, где  $n$ - целое?\footnote{Для решения этой задачи дети должны уметь выносить общий множитель на скобку. Если по какой-то причине это еще не изучено, то лучше давать задачу в виде произведения $ n $ и $ ( n+1) $}
\end{thm}

\begin{thm}$^n$\label{1.4} Может ли для каких-нибудь целых чисел $a$ и $b$ быть верно:    $$ab(a-b) = 201720182019$$
\end{thm}

%\begin{prf}		\textbf{1 способ.} Рассмотрим два случая. 1 случай: числа $a$ и $b$ одной четности. Тогда $a - b$ обязательно четно. Следовательно, произведение $ab(a-b)$ также будет четным. 2 случай: числа a и b разной четности. Тогда в произведении $ab(a-b)$ один из множителей четен и, следовательно, все произведение четно. Тем самым мы доказали. Что для любых целых чисел а и b произведение $ab(a-b)$ четно. Но число 200920102011 нечетно, следовательно, требуемое в условии невозможно.\\
%\textbf{2 способ.} Поскольку 200920102011 - нечетное число, то требуется выяснить, можно ли его разложить на три нечетных множителя указанного в условии вида. Предположим, что это возможно, тогда числа $a$ и $b$ должны быть нечетными, но тогда $a - b$ обязательно четно. Следовательно, произведение $ab(a-b)$ также будет четным. Противоречие. Следовательно, требуемое в условии невозможно.
%		\end{prf}\\

В предыдущих задачах у нас обычно имелось фиксированное количество целых чисел, с которыми мы производили некоторые операции. Для решения задачи требовалось выяснить, каким - четным или нечетным - числом является результат этих операций. Можно рассмотреть обратную задачу. Пусть имеется некоторое целое число, и мы хотим разбить его на части. Возникает вопрос: на какие части мы можем его разбить и сколько среди них может быть нечетных? Или, другими словами, в виде суммы каких слагаемых можно представить данное число? Например, число 17 можно представить в виде суммы трех или пяти нечетных слагаемых, но нельзя в виде четырех.

\begin{center}
	\large\textbf{Количество нечетных чисел.}
\end{center}

\begin{thm}
	Сумма четырнадцати целых чисел является нечетным числом. Может ли их произведение тоже быть нечётным?
\end{thm}

%\begin{thm}Произведение 22 целых чисел равно 1. Может ли их сумма равняться нулю?
%\end{thm}

\begin{thm}
	За время летних каникул вдоль забора школы посадили 20 яблонь. 1 сентября оказалось, что число яблок на соседних деревьях отличается на 1. Может ли на всех этих яблонях быть ровно 2011 яблок?
\end{thm}

\begin{thm}\label{1.7}
	На доске написано 2011 целых чисел. Всегда ли можно стереть одно из них так, чтобы сумма всех оставшихся чисел была четна?
\end{thm}
\begin{thm}\label{1.8}	Алиса и Базилио устроили благотворительную лотерею. Они написали на 33 билетах (занумерованных числами от 1 до 33) 33 последовательных числа от 33 до 65 (в каком-то порядке) и объявили, что выигрышным считается билет, у которого сумма номера билета и написанного на нем числа четна. Докажите, что при таких правилах им в любом случае придется кому-то выплатить выигрыш.
\end{thm}
\begin{thm}\label{an4.1}	98 спичек разложили в 19 коробков и на каждом написали количество спичек в этом коробке. Может ли произведение этих чисел быть нечетным числом? Если да, то приведите пример, если нет, то докажите, почему.
\end{thm}

Выполняя упражнение \ref{u7} из теоретического листочка, вы, несомненно, воспользовались тем, что четные и нечетные числа в числовом ряду идут попеременно:

\begin{table}[h]\centering
	\begin{tabular}{|c|c|c|c|c|c|c|}
		\hline
		1&2&3&4&5&6&...\\
		\hline
		Н&Ч&Н&Ч&Н&Ч&...\\
		\hline
	\end{tabular}
\end{table}

В следующих задачах вам придется пользоваться этим соображением неоднократно. Более того, идея чередования характерна не только для чисел. Зачастую чередуются не числа, а какие-либо свойства объектов.

\begin{thm}	\label{an3.2}
	Можно ли разложить несколько мячей а)$^\ast$ в 3; б)$^\ast$ в 4; в) в 2018; г) в 2019 ящиков, расставленных по кругу, так, чтобы в любых двух соседних ящиках число мячей отличалось на 1?
\end{thm}

\begin{thm}\label{1.9}
	Вокруг круглой поляны растут 2011 сосен. Незнайка измерил высоту каждой из них и заявил, что любые две соседние сосны отличаются по высоте ровно на метр. Знайка тут же заметил, что Незнайка врет. Кому верить?
\end{thm}

\begin{thm}\label{an5.1}
	Можно ли обойти конем всю доску, побывав на каждом поле ровно один раз, начав с поля а1, а закончив на поле h8? (Если можно, то как, если нельзя, то почему.)
\end{thm}

\newpage