\head{Ноябрь}{Листок 4. Теория чисел. Уровень 2.}

\begin{thm}
    Докажите, что число не может попасть в два разных класса (т.е. что число не может давать двух различных остатков при делении на одно и то же число).\footnote{Другими словами мы просим вас доказать корректность определения деления с остатком.}
\end{thm}
    
\begin{thm}
    Можно ли разрезать 6 прутьев длиной по 1 м каждый на 10 кусков длиной 27 см, 16 кусков длиной 15 см и 15 кусков длиной 6 см?
\end{thm}

\begin{thm}
    Докажите, что при любом натуральном $n$ число $n^3 + 3n^2 + 2n$ кратно 6.
\end{thm}

\begin{thm}
    Докажите, что при любом натуральном $n$ число $n(n + 1)^2(3n + 2)$ кратно 4.
\end{thm}

\begin{thm}
    Докажите, что для любого простого $р > 3$ число $р^2 – 1$ делится на 24.
\end{thm}

\begin{thm} $^n$ \label{4.2 thm1}
         Докажите, что при любом натуральном $n$ число $n^3 + 5n$  делится на 3. 
\end{thm}  

\begin{center}
Для решения следующих двух задач рекомендуется вспомнить задачи из листка 2. 
\end{center}

\begin{thm} Докажите, что
    \par 
    а)~из 8 целых чисел всегда можно выбрать два таких, разность которых делится на 7. 
    \par 
    б)~из 5 чисел всегда можно выбрать два таких, у которых разность квадратов делится на 7.
\end{thm}

\begin{thm}
    Докажите, что среди чисел, написанных только единицами найдётся число,
    \\
    делящееся на 2012.
\end{thm}

\begin{thm}
    Докажите, что сумма квадратов двух последовательных целых чисел при делении на 4 даст остаток 1.
\end{thm}

\begin{thm}
    Докажите, что, если число $a^2 + b^2$ делится на 7, то числа $a$ и $b$ делятся на 7.
\end{thm}

\begin{thm} $^n$ \label{4.2 thm2} 
    Число $а$ – чётное, не кратное 4. Докажите, что число $а^2$ при делении на 32 даст остаток 4.
\end{thm}

\begin{thm}
    Число $а$ не делится ни на 2, ни на 3. Найдите остаток от деления числа $а^2$ на 6.
\end{thm}

\fbox{\begin{minipage}{0.95\textwidth}
\begin{thm}
    Докажите, что при любом целом $a$ число $a^2 + 1$ не делится на 3.
\end{thm}
\end{minipage}} 

\begin{center}
Предыдущая задача представляется нам очень важной, поэтому она выделена в тексте \\ и несколько следующих задач используют её же идею. 
\end{center}

\begin{thm}
    Целые числа $х, у, z$ таковы, что $х^2 + у^2 = z^2$. Докажите, что хотя бы одно из этих чисел делится на 3.
\end{thm}

\begin{thm} $^n$ \label{4.2 thm3}
    Может ли сумма квадратов двух целых чисел, не кратных 3, быть квадратом некоторого целого числа?
\end{thm}

\begin{thm}
    Докажите, что остаток от деления натурального числа на 3 (или 9) равен остатку от деления на 3 (или, соответственно, на 9) суммы его цифр.
\end{thm}

\begin{thm}
    Сформулируйте и докажите признаки делимости а)~на 3;~б)~на 9;~в)~на 11.
\end{thm}

\begin{thm}
    Из трёхзначного числа вычли число, получающееся из него же перестановкой цифр. Докажите, что результат вычитания делится на 9.
\end{thm}

\begin{thm} 
     Ваня показывает числовой фокус. Задумайте трёхзначное число с разными цифрами. Запишите его цифры в обратном порядке. Получится ещё одно число. Вычтите из большего числа меньшее. Зачеркните в полученной разности любую цифру, кроме нуля, а оставшееся число сообщите Ване. Ваня тут же назовет вычеркнутую цифру. Как он это делает?
\end{thm}

\begin{thm}
    Найдите все пятизначные числа 
    \par
    а)~вида $\overline{34x5y}$, которые делятся на 36; б)~вида $\overline{71x1y}$, которые делятся на 45. 
\end{thm}

\textbf{\textit{Внимание! На этот раз в этой части три письменных задачи – \ref{4.2 thm1}, \ref{4.2 thm2} и \ref{4.2 thm3}.}}

