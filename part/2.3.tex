\head{Октябрь}{Листок 1. Принцип Дирихле}
\epigraph{\textit{Если ты в своем кармане ни копейки не нашел, загляни в карман к соседу - очевидно, деньги там!}}{Григорий Остер "Вредные советы"}
\begin{thm}$^\circ$
	В ящике лежат шары: 5 красных, 7 синих и 1 зеленый. Сколько шаров надо вынуть, не глядя, чтобы наверняка достать 2 шара одного цвета?
\end{thm}

\begin{thm}
	На Марс привезли 51 ящик с приборами пяти видов. Верно ли, что обязательно приборов какого-то вида было а) не более 10 ящиков, б) не менее 10 ящиков?
\end{thm}

\begin{thm}\label{2.3}
	$^n$ Вася в течение 3 дней съел 100 шоколадок. Обязательно ли найдется ли день, в который Вася съел а) не менее 34 шоколадок б) не более 33 шоколадок?
\end{thm}

%\begin{prf}
%	а) Предположим противное: такого дня не найдется. Это означает, что в каждый из трех дней Вася съедал менее 34 шоколадок, то есть 33 или менее. Тогда за три дня он мог съесть максимум $3\times33=99$ шоколадок, а он съел больше -- 100. Противоречие. Следовательно, наше предположение неверно и такой день найдется.
%	
%	б) Предположим противное: такого дня не найдется. Это означает, что в каждый из трех дней Вася съедал более 33 шоколадок, то есть 34 или более. Тогда за три дня он мог съесть минимум $3\times34=102$ шоколадок, а он съел меньше -- 100. Противоречие. Следовательно, наше предположение неверно и такой день найдется.
%\end{prf}

\begin{thm}
	а) В темной комнате стоит шкаф, в ящике которого лежат 24 черных и 24 синих носка. Сколько носков следует взять из шкафа, чтобы из них заведомо можно было составить по крайней мере одну пару носков одного цвета? (речь идет о минимальном числе носок)	
	б) Сколько надо взять носков, чтобы заведомо можно было составить хотя бы одну пару носков черного цвета?	
	в) Как изменится решение задачи, если в ящике лежат 12 пар черных и 12 пар синих ботинок и требуется составить пару одного цвета (как в пункте а) и пару черного цвета (как в пункте б)?
\end{thm}

\begin{thm}
	Белых и черных кроликов рассаживали по клеткам. Белых было больше, чем черных. Докажите, что найдется клетка, в которой белых кроликов больше, чем черных.
\end{thm}

\begin{thm}
	В коробке лежат карандаши: 7 красных и 5 синих. В темноте берут карандаши. Сколько надо взять карандашей, чтобы среди них было не меньше 2-х красных и не меньше 3-х синих? 
\end{thm}

\begin{thm}
	В ящике лежат цветные карандаши: 10 красных, 8 синих и 4 желтых. В темноте берем из ящика карандаши. Какое наименьшее число карандашей надо взять, чтобы среди них заведомо было:\\
	а) не менее 4-х карандашей одного цвета?~~
	б) не менее 6-ти карандашей одного цвета?\\
	в) хотя бы 1 карандаш каждого цвета?~~~~~~
	г) не менее 6-ти синих карандашей? 
\end{thm}

\begin{thm}
	Докажите, что среди любых $n$ чисел найдутся два, разность\footnote{Здесь вам понадобится понятие остатка при делении целых чисел.} которых делится на $n-1$. 
\end{thm}

\begin{thm}
	В классе 40 человек. В диктанте Степа сделал 13 ошибок, а остальные сделали меньше. Обязательно ли найдутся ли среди них а) 3 человека, б) 4 человека, сделавших одинаковое число ошибок?
\end{thm}

\begin{thm}
	У человека на голове не более $400000$ волос, в Москве более 8 млн. жителей. Докажите, что найдутся 20 москвичей с одинаковым числом волос.
\end{thm}

\begin{thm}
	О населении города Поданк известно следующее:\\
	а) Среди жителей Поданка не найдется двух с равным числом волос на голове.\\
	б) Ни у одного жителя Поданка не голове не растет ровно 518 волос.\\
	в) Жителей в Поданке больше, чем волос на голове любого из них.\\
	Какова может быть наибольшая численность населения г. Поданка?
\end{thm}
\textit{Напоминаем, что для получения задач следующего уровня необходимо выполнить проверочную работу, включающую задачи, похожие на приведенные выше. Задачи, отмеченные значком п, приниматься в устном форме не будут. В данном случае это единственная задача \ref{2.3}. Решение таких задач надлежит выполнить в письменном виде дома и принести на занятие спецматематики.
	Рекомендуем решать задачи по порядку, поскольку зачастую в решении следующей задачи можно использовать идею из предыдущей.
	\begin{flushright}
		Желаем успеха!
	\end{flushright}
}

\head{Октябрь}{Листок 1. Принцип Дирихле. Проверочная работа.}
\begin{thm}
	Докажите, что среди любых шести человек всегда найдутся либо трое попарно знакомых, либо трое попарно незнакомых.
\end{thm}

\begin{thm}
	Каждый из 17 ученых переписывается с остальными. В их переписке речь идет лишь о трех темах. Каждая пара ученых переписывается друг с другом только по одной теме. Докажите, что не менее трех ученых переписываются друг с другом по одной и той же теме.
\end{thm}

\begin{thm}
	В погребе стоит 20 одинаковых банок с вареньем. В 8-ми банках клубничное варенье, в 7-ми - малиновое, в 5-ти - вишневое. Каково наибольшее число банок, которые можно в темноте вынести из погреба с уверенностью, что там осталось еще хотя бы 4 банки одного сорта варенья и 3 банки другого? 
\end{thm}

\begin{thm}
	За круглым столом сидят 25 детей и 25 преподавателей и грустно смотрят друг на друга.
	а) Докажите, что есть ребенок, который сидит точно напротив преподавателя.
	б) Докажите, что есть несчастный, оба соседа которого - дети.
\end{thm}
\begin{thm}
	$^{\ast\ast}$В классе 25 человек. Известно, что среди любых трех из них есть двое друзей. Докажите, что есть ученик, у которого не менее 12 друзей.
\end{thm}

\head{Октябрь}{Листок 2. Принцип Дирихле.}

\epigraph{ \textit{«В три клетки нельзя посадить четырех зайцев так, чтобы в каждой клетке сидел ровно один заяц. Это следует из принципа Дирихле\dots} \\ \textit{Заменим ученых кроликами и посадим их в ящики\dots}»}{Из школьных записей}
В предыдущих задачах было ясно из условия, что именно надо считать “клетками”, а что - “кроликами”. Обычно угадать “кроликов” несложно. Скорее всего, это те объекты, про которые и надо что-то выяснить, хотя это и не всегда. Иногда надо рассматривать не все предложенные объекты, а только часть из них, или, наоборот, добавить к ним еще несколько и рассматривать все полученное множество. Перед решением каждой из следующих задач мы попросим вас указать, что именно вы считаете «клетками», а что - «кроликами»

\begin{thm}
	Докажите, что равносторонний треугольник нельзя покрыть двумя меньшими равносторонними треугольниками.
\end{thm}

\begin{thm}\label{2.13}
	$^n$ В спичечных коробках находятся 50 тараканов. Докажите, что либо в одном коробке живут 8 тараканов, либо для выкидывания коробков по одному в день потребуется больше недели.
\end{thm}
%
%\begin{prf}
%	Предположим противное: в каждом коробке живут не более 7 тараканов и таких коробков не более семи (поскольку в неделе семь дней). Тогда в этих коробках живет максимум 7х7=49 тараканов, но у нас их больше. Противоречие. Следовательно, наше предположение неверно и либо в одном коробке живут 8 или более тараканов, либо для выкидывания коробков по одному в день потребуется больше недели.
%\end{prf}

\begin{thm}
	По краю круглого стола расставлены таблички с фамилиями дипломатов, участвующих в переговорах о борьбе с мировым терроризмом. Оказалось, что дипломаты не посмотрели на таблички и каждый сел не на свое место. Можно ли повернуть стол так, чтобы хотя бы два дипломата сидели против своих табличек?
\end{thm}

\begin{thm}
	$^\ast$Докажите, что среди чисел, записываемых одними единицами, найдется число, делящееся на 2007.
\end{thm}

\begin{thm}
	$^\ast$Карабас Барабас пообещал Буратино открыть тайну Золотого ключика, если тот составит из чисел 0, 1, 2 волшебный квадрат 6х6 так, что в каждой строке, каждом столбце и двух диагоналях сумма чисел была различна. Помогите Буратино!
\end{thm}

\begin{thm}
	$^\ast$Дано 11 различных натуральных чисел, не превосходящих 20. Докажите, что из них можно выбрать два, одно из которых делится на другое. 
\end{thm}

\begin{thm}
	Докажите, что среди любых трех целых чисел найдется два, сумма которых четна.
\end{thm}

\begin{thm}
	$^{\ast\ast}$ Докажите, что среди любых 10 целых чисел найдутся одно или несколько, сумма которых делится на 10.
\end{thm}

\begin{thm}
	Докажите, что среди любого количества человек найдутся двое, имеющих одинаковое (возможно, нулевое) число знакомых среди этих человек.
\end{thm}

\begin{thm}
	Александр Кириллович предложил ученикам 7Ю класса написать два теста, за каждый из которых можно было получить оценку 1, 2, 3, 4 или 5. Верно ли, что обязательно найдутся два ученика получившие одинаковые оценки по обоим тестам, если в 7Ю учится 26 человек?
\end{thm}

\begin{thm}
	В ковре размером 4х4 метра моль проела 15 дырок. Докажите, что из него можно вырезать коврик размером 1х1 метр, не содержащий внутри себя дырок. (Дырки считайте точечными.)
\end{thm}

\textit{Напоминаем, что получения задач следующего уровня необходимо выполнить проверочную работу, включающую задачи, похожие на приведенные выше. Задачи, отмеченные значком п, приниматься в устном форме не будут. В данном случае это единственная задача \ref{2.13}. Решение таких задач надлежит выполнить в письменном виде дома и принести на занятие спецматематики.
	Рекомендуем решать задачи по порядку, поскольку зачастую в решении следующей задачи можно использовать идею из предыдущей.
	\begin{flushright}
		Желаем успеха!
	\end{flushright}
}

\head{Октябрь}{Листок 2.2. Принцип Дирихле. Проверочная работа. Аналогии.}

И снова аналогии! В это раз в этом разделе нет оригинальных задач. Поэтому решать их вам не придется. Для выполнения этого задания вам нужно только найти задачи, аналогичные приведенным выше и описать соответствие.

\begin{thm}
	В кружок ходит 17 школьников. На дом им задали 3 задачи. Как потом оказалось, каждые два школьника посоветовались друг с другом ровно по одной задаче. Докажите, что найдется три кружковца, посоветовавшиеся попарно друг с другом по одной и той же задаче.
\end{thm}

\begin{thm}
	На бумаге нарисованы шесть точек. Каждые две из них соединены либо красным отрезком, либо синим. Докажите, что найдется либо красный треугольник с вершинами в нарисованных точках, либо синий.
\end{thm}

\begin{thm}
	В соревнованиях по вольной борьбе участвовало 12 человек. Каждый участник должен был встретиться с каждым из остальных по одному разу. Докажите, что в любой момент соревнования имеются два участника, проведшие одинаковое число схваток.
\end{thm}

\begin{thm} \label{2.3 thm2.33}
	Мальчики танцевали вальс с девочками. После танца девочки встали в круг, и к каждой из них подошел мальчик. Оказалось, что не составилась ни одна танцевальная пара. Могут ли мальчики так перейти по кругу, чтобы составились хотя бы две танцевальные пары?
\end{thm}

\begin{thm}
	В одной пещере живут шесть гномов. Каждый день каждые два из них либо ссорятся либо снова мирятся. Докажите, что в любой день найдутся либо трое попарно помирившихся гнома, либо трое попарно поссорившихся.
\end{thm}

\begin{thm}
	17 кумушек обсуждают по телефону три темы. Причем любые две обсуждают всегда только одну неизменную тему. Докажите, что найдутся трое, обсуждающие одну и ту же тему.
\end{thm}

\begin{thm}
	В классе 25 человек. Докажите, что обязательно найдутся два человека, имеющих одинаковое количество друзей в этом классе.
\end{thm}

\begin{thm}
	Среди шести бульбозавров любые два могут договориться друг с другом на одном из двух языков: на бульбо-языке или на завро-языке. Докажите, что найдутся три бульбозавра, говорящих на одном и том же языке.
\end{thm}
\head{Октябрь}{Листок 3. Принцип Дирихле.}
\begin{thm}
	Докажите, что среди чисел, записываемых одними единицами, найдется число, делящееся на 2011.
\end{thm}

\begin{thm}
	Карабас Барабас пообещал Буратино открыть тайну Золотого ключика, если тот составит из чисел 0, 1, 2 волшебный квадрат 6х6 так, что в каждой строке, каждом столбце и двух диагоналях сумма чисел была различна. Помогите Буратино!
\end{thm}

\begin{thm}
	Дано 11 различных натуральных чисел, не превосходящих 20. Докажите, что из них можно выбрать два, одно из которых делится на другое. 
\end{thm}



До сих пор мы рассматривали применение принципа Дирихле для дискретных величин (то есть величин, которые можно сосчитать). Однако этот принцип точно так же можно использовать и для непрерывных величин (то есть величин, которые можно измерить). Например:

\begin{table}[h]
	\centering	
	\begin{tabular}{|c|}
		\hline
		\textit{Если $N$ кроликов съели $M$ кг травы,} \textit{то какой-то кролик съел не меньше $\dfrac{M}{N}$ кг }\\
		\textit{и какой-то съел не больше $\dfrac{M}{N}$ кг.}\\(А если кто-то съел больше среднего, то кто-то съел меньше среднего)\\\hline
	\end{tabular}
\end{table}
Заметим, что в последней формулировке кролики играют роль клеток, а трава - роль кроликов, сидящих в этих клетках.

\begin{thm}
	Пусть сумма $N$ чисел равна $A$. Докажите\footnote{Доказательство в этой задаче как раз и является доказательством непрерывного принципа Дирихле.}, что среди этих чисел найдется число  а) не больше $\dfrac{A}{N}$ ~~~ б) не меньше $\dfrac{A}{N}$ .
\end{thm}

\begin{thm}
	$^\circ$Семья из семи человек ела торт. Его разделили на 7 частей разного размера. Докажите, что кто-то съел не меньше 1/7 торта.
\end{thm}

\begin{thm}
	Четырехугольник площади S четырьмя прямыми разбили на 9 четырехугольников. Докажите, что найдется четырехугольник, площадь которого не меньше S/9. \footnote{Огромная просьба: не нужно сводить решение этой задачи к фразе: «все следует из принципа Дирихле». Хотелось бы услышать доказательство.}
\end{thm}

\begin{thm}
	На плоскости проведено 12 прямых, проходящих через одну точку. Докажите, что какие-то две из них образуют угол не больше 15$^\circ$.
\end{thm}

\begin{thm}
	Докажите, что у часов с часовой, минутной и секундной стрелкой в любой момент времени найдутся стрелки, угол между которыми не больше 120$^\circ$.
\end{thm}

\begin{thm}
	$^\ast$На отрезке длиной 20 см отмечено 19 точек. Докажите, что найдется интервал длиной не меньше 1 см , не содержащей ни одной выделенной точки.\footnote{интервал - это «отрезок без концов»}
\end{thm}

\begin{thm}
	$^\ast$На планете Зям-Лям в звездной системе Тау Кита суша занимает больше половины площади всей планеты. Докажите, что таукитяне могут прорыть туннель через центр планеты и соединяющий сушу с сушей.\footnote{Будем считать, что техника у них для этого достаточно развита, а планета имеет форму шара.}
\end{thm}

\begin{thm}\label{2.29}
	$^{\ast\ast}$Пусть в предыдущей задаче планета не обязательно имеет форму шара. Верно ли тогда утверждение про туннель?
\end{thm}

В предыдущих задачах только стоило догадаться, к чему именно надо применить принцип Дирихле и задача “решалась”. Но часто в одной задаче приходится применять этот принцип не один раз, а то и применять, используя разные “клетки”, т.е. разные способы разбиения на группы предметов, описываемых в задаче. 

\textit{В этот раз уровень 3 является не последним. Выполнение задач этого уровня дает возможность получить 4+ и возможность сдавать задачи четвертого уровня. Задача \ref{2.29} не является обязательной для решения. \\Желаем успеха!}

