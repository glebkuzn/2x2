\head{Октябрь}{Листок 2.4. Принцип Дирихле. Проверочная работа.}
\begin{thm}
	$^{\ast}$Докажите, что среди любых 10 целых чисел найдутся одно или несколько, сумма которых делится на 10.
\end{thm}

\begin{thm}
	$^{\ast\ast}$Две вершины выпуклого 2005-угольника будем называть далекими, если диагональ, соединяющая их, делит этот 2005-угольник на 1004-угольник и 1003-угольник. Докажите, что если все вершины исходного 2005-угольника раскрасить в 3 цвета, то найдутся либо 2 одноцветные соседние вершины, либо 2 одноцветные далекие вершины.
\end{thm}

\begin{thm}
	$^{\ast}$Какое наименьшее количество карточек «Спортлото – 6 из 49» нужно купить, чтобы наверняка хоть на одной из них был угадан хоть один номер? \footnote{Игра «Спортлото» состоит в том, что в таблице чисел от 1 до 49 следует зачеркнуть 6 любых номеров, после чего происходит розыгрыш – объявляются выигрышными какие-то 6 номеров, если происходит совпадение – этот номер угадан.}
\end{thm}

\begin{thm}
	$^{\ast}$Каждая клетка таблицы 2007х2007 покрашена в один из 2006 цветов. За ход можно взять строку или столбец и, если там есть две клетки одного цвета, перекрасить эту строку или столбец в этот цвет. Можно ли за несколько ходов покрасить всю таблицу в один цвет?
\end{thm}

\begin{thm}
	$^{\ast}$В клетках шахматной доски 8 х 8 записаны числа 1, 2, 3, ..., 62, 63, 64. Докажите, что найдутся две такие соседние клетки, что числа, записанные в них, отличаются не меньше, чем на 9. (Соседними считаются клетки, имеющие общую сторону или вершину).
\end{thm}

\begin{thm}
	$^{\ast\ast}$В лесу живут 20 гномов, каждый из которых дружит по крайней мере с 14-ю другими гномами. Обязательно ли найдутся 4 гнома, которые все дружат между собой?
\end{thm}

\begin{thm}
	$^{\ast\ast}$На столе лежат 50 правильно идущих часов со стрелками. Докажите, что в некоторый момент сумма расстояний от центра стола, до концов минутных стрелок будет больше, чем сумма расстояний от центра стола до центров часов.
\end{thm}

\begin{thm}
	$^{\ast\ast}$Комиссия из 60 человек провела 40 заседаний, причем на каждом присутствовало ровно 10 членов комиссий. Докажите, что какие-то два члена комиссии встречались на её заседаниях по крайней мере дважды.
\end{thm}

\begin{thm}
	$^{\ast\ast}$На плоскости отмечены 6 точек так, что любые три из них образуют треугольник со сторонами разной длины. Докажите, что найдутся два треугольника таких, что наименьшая сторона первого одновременно является наибольшей стороной второго.
\end{thm}

\begin{thm}
	$^{\ast\ast}$	Узлы бесконечной клетчатой бумаги раскрашены в два цвета. Докажите, что существуют две вертикальные и две горизонтальные прямые, на пересечении которых лежат точки одного цвета.
\end{thm}
\newpage

\section{Ответы к упражнениям} \textbf{\ref{u12}}	а) … не менее двух конфет; б) не менее двух привидений; в) не менее трех монеток.\\ \textbf{\ref{u13}.}	всегда верны только б) и е).
\section{Решения некоторых задач}
\textbf{Задача \ref{2.13}}
	$^n$ В спичечных коробках находятся 50 тараканов. Докажите, что либо в одном коробке живут 8 тараканов, либо для выкидывания коробков по одному в день потребуется больше недели.

\begin{prf}
	Предположим противное: в каждом коробке живут не более 7 тараканов и таких коробков не более семи (поскольку в неделе семь дней). Тогда в этих коробках живет максимум 7х7=49 тараканов, но у нас их больше. Противоречие. Следовательно, наше предположение неверно и либо в одном коробке живут 8 или более тараканов, либо для выкидывания коробков по одному в день потребуется больше недели.
\end{prf}

\textbf{Задача \ref{2.3}}
	$^n$ Вася в течение 3 дней съел 100 шоколадок. Обязательно ли найдется ли день, в который Вася съел а) не менее 34 шоколадок б) не более 33 шоколадок?

\begin{prf}
	а) Предположим противное: такого дня не найдется. Это означает, что в каждый из трех дней Вася съедал менее 34 шоколадок, то есть 33 или менее. Тогда за три дня он мог съесть максимум $3\times33=99$ шоколадок, а он съел больше -- 100. Противоречие. Следовательно, наше предположение неверно и такой день найдется.
	
	б) Предположим противное: такого дня не найдется. Это означает, что в каждый из трех дней Вася съедал более 33 шоколадок, то есть 34 или более. Тогда за три дня он мог съесть минимум $3\times34=102$ шоколадок, а он съел меньше -- 100. Противоречие. Следовательно, наше предположение неверно и такой день найдется.
\end{prf}