\head{Февраль}{Листок 5. Теория множеств. Уровень 3.}

\begin{thm}
Каждый из учеников класса занимается не более чем в двух кружках, причём для любой пары учеников существует кружок, в котором они занимаются вместе. Докажите, что найдётся кружок, в котором
занимаются не менее $\dfrac{2}{3}$ этого класса.
\end{thm}

\begin{thm}
Дано множество $M$ из $n$ элементов, в котором выбрано несколько подмножеств. Известно, что любое не выбранное подмножество данного множества $M$ представляется в виде пересечения некоторых выбранных подмножеств. Какое наименьшее число подмножеств могло быть выбрано?
\end{thm}

\begin{thm}$^{\ast}$
\textit{(Теорема Холла о различных представителях.)} Пусть система множеств такова, что объединение любых $k$ из них содержит не менее $k$ различных элементов. Докажите, что тогда можно выбрать по одному элементу в каждом множестве так, что все выбранные элементы будут различны.
\end{thm}

\begin{thm}$^{\ast}$
Дано несколько различных натуральных чисел. Известно, что среди любых трёх из них можно выбрать два так, чтобы одно делилось на другое. Докажите, что все числа можно разбить на две группы так, чтобы для любых двух чисел из одной группы одно число делилось на другое.
\end{thm}

\begin{thm}$^{\ast}$
Среди бесконечного количества гангстеров каждый охотится за каким-нибудь одним из остальных. Докажите, что существует бесконечное подмножество этих гангстеров, в котором ни один не охотится за кем-либо из этого подмножества.
\end{thm}

\begin{thm}$^{\ast}$
Даны натуральные числа $n \geq k \geq 1$. Рассмотрим все возможные подмножества множества $\{1, 2, ... , n\}$, состоящие из $k$ чисел, и в каждом из них выберем наименьшее. Докажите, что среднее арифметическое всех выбранных чисел равно $\dfrac{n+1}{k+1}$. (Например, $n = 3$, $k = 2$. Исходное множество есть $\{1, 2, 3\}.~k$ - элементные подмножества есть $\{1, 2\}$, $\{1, 3\}$, $\{2, 3\}$. Выбираем в каждом из них наименьший элемент, получаем:
1, 1, 2. Среднее арифметическое равно $\dfrac{1+1+2}{3} = \dfrac{4}{3} = \dfrac{n+1}{k+1}$)
\end{thm}