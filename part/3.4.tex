\head{Ноябрь}{Листок 3. Комбинаторика.}

\begin{thm}
	$^{\ast}$ Сколькими способами можно представить число $n$ в виде нескольких натуральных слагаемых, если а) $n = 3$;~~ б) $n = 33$;~~ в) $n$ - произвольное натуральное число?
\end{thm}

\begin{thm}
	$^{\ast}$ Сколькими способами можно представить
	а) число 11 в виде 3 целых положительных слагаемых?~~~~
	б) число n в виде 3 целых положительных слагаемых?\\
	в$^{\ast\ast}$) число n в виде  $k < n$  целых положительных слагаемых?\footnote{Способы, отличающиеся порядком слагаемых, считаются различными}
\end{thm}

\begin{thm}
	$^{\ast}$ Сколькими способами можно расставить на шахматной доске восемь\\ а) разноцветных~~~  б) черных ладей, чтобы они друг друга не били?
\end{thm}

\begin{thm}
	$^{\ast}$ Докажите формулу: $(n + 1)! - n! = n!\times n$
\end{thm}

\begin{thm}
	$^{\ast}$ Сколькими способами можно поставить на шахматную доску\\ а) белого и черного~~~ б) двух белых королей так, чтобы они не били друг друга?
\end{thm}

\begin{thm}$^{\ast\ast}$
	\textquotedblleft Распался коллектив музыкального ансамбля племени Мумба-Бумба! На седьмом костре состоится формирование нового музыкального ансамбля из 37 аборигенов!\textquotedblright - звучали Мумб-Бумбские новости, переведенные лингвистами на наш язык. Также ученые выяснили, что в племени Мумба-Бумба были аборигены 7 различных расцветок (в каждой деревне жили аборигены своей расцветки, это как разные флаги в разных странах): Белые, Черные, Красные, Желтые, Синие, Зеленые и Серо-буро-малиновые в крапинку (или в горошек, ученые точно не установили). Возникает весьма интересный и высоко научный вопрос: сколько предстояло тогда Мумб-Бумбскому жюри перебрать различных вариантов составления нового ансамбля, если считать, что аборигены одной расцветки все на одно лицо и что\\
	а) в ансамбле должны присутствовать аборигены всех расцветок?\\
	б) допускается отсутствие в ансамбле аборигенов какой-либо расцветки?
\end{thm}

\begin{thm}
	Каких семизначных чисел больше: тех, в записи которых есть семерка или остальных?
\end{thm}

\begin{thm}$^{\ast\ast}$
	Сколькими способами можно разбить 7А класс (25 человек) на две не обязательно равные группы?
\end{thm}

\begin{thm}$^{\ast\ast}$
	В левом нижнем углу шахматной доски стоит ладья. Сколькими способами она может пройти в правый верхний угол, если она может ходить только вверх и вправо?\footnote{Одними и теми же способами считаются те, когда их пути совпадают.}
\end{thm}
\newpage 

\section{Ответы к упражнениям}
\textbf{\ref{u14}.}	а) 35; б) 41.~~ \textbf{\ref{u15}.} на 105 лет.~~ \textbf{\ref{u16}.} Например, так: из города А в город Б ведет 3 дороги, из города Б в город В 5 дорог, а из города В в город Г - 7 дорог. Сколькими способами можно проехать из города А в Г?~~\textbf{ \ref{u17}.} $2^{100}$.~~ \textbf{\ref{u18}.} Между городами А и В, В и С и С и Д по две дороги. Сколькими способами можно проехать из А в Д?~~ \textbf{\ref{u19}.} а) 20!; б) 8!;  в) $(N+1)!$~~  \textbf{\ref{u20}.} а) $ \dfrac{101!}{99!}=\dfrac{99!\times 100\times 101}{99!} = 100\times 101 = 10100$; б) $\dfrac{N!}{(N-1)!} =\dfrac{(N-1)!\times N}{(N-1)!} = N$.~~ \textbf{Задача \ref{3.6}}  $P_N=N!$
\section{Решения некоторых задач}
\textbf{Задача \ref{3.17}}
	На бумажной полоске написано слово АБВГДЕЙКА. Сколькими способами можно разрезать эту полоску на три слова? \footnote{Не обязательно осмысленных}

\begin{prf}
	Резать полоску можно только между буквами, поэтому по сути дела задача сводится к выбору промежутков между букв, где нужно поставить перегородки. Так как требуется разрезать на три слова, то надо выбрать два промежутка. Букв всего девять, следовательно, промежутков восемь. Таким образом, задача свелась к задаче о том, сколькими способами можно выбрать два предмета из восьми. Первый предмет можно выбрать восемью способами, а второй уже только семью. Всего $8\times7=56$ способов. Но каждый из способов мы сосчитали дважды (например, выбор предмета №6 и №8 можно осуществить так: выбрать сначала №6, а потом №8, или наоборот - сначала №8, а затем №6). Поэтому мы должны полученный результат разделить на два. 
 
 Итак, окончательный ответ: 28 способов.
\end{prf}