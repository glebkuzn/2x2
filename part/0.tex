
		\section{Предисловие}

	Эта книжка предназначена в первую очередь для учителей математики, работающих в профильных математических классах. Материалы книги могут быть также использованы в работе математических кружков, для дополнительных занятий и самостоятельной работы.
	
	Мы будем ориентироваться на учителей математики, представляя удобный формат занятий в первую очередь для них.
	
	Сама структура занятий основана на «листочковой системе», введенной в школьное образование Н.Н.Константиновым еще в 60е годы прошлого века. Начав заниматься спецматематикой с более младшими школьниками, нежели те, для которых изначально была разработана эта система, мы переработали и сам набор и порядок тем, и саму структуру.
	 
	Система была опробована автором в течение 15 лет на различных математических классах г.Москвы, выпускники которых завоевали четыре золотых и две серебряные медали на Международной математической олимпиаде школьников и получили множество наград на других всероссийских и международных соревнованиях. Часть из них уже закончила свое обучение в Вузе и плодотворно трудится на научном поприще.
	
	В течение этих лет система модифицировалась, пока не пришла к нынешнему виду. Сейчас она достаточно гибка и, на наш взгляд, позволяет подстроиться к любому уровню школьников 7 класса, желающих изучать профильную математику.
	
	По сути, система представляет собой «Уровневую систему листочков» \footnote{ Напомним, что «система в листочках» предполагает, что в классе находится несколько принимающих, что дает возможность каждому индивидуально рассказывать решения задач принимающим. Кроме того на занятиях не предполагаются учебники. Все необходимые сведения выдаются в листках и дети могут решать эти задачи, как в классе, так и дома. Тем самым каждый имеет свой собственный ритм работы и движения. Одномоментно в классе дети могут иметь много разных листков и сдавать задачи в индивидуальном темпе.}, правила которой объявляются детям заранее.	Лучше всего эти правила распечатать и выдать детям. Вот они:
	\begin{center}
		\textbf{Правила игры или как мы будем жить.}
	\end{center}
\begin{enumerate}
	\item Задания по спецматематике сгруппированы по темам, а в каждой теме разнесены на два, три или более уровня.
	\item Темы бывают обязательные и дополнительные. 
	\item Изначально при изучении новой темы все получают теоретический листок и листок уровня «один». Усвоение теоретического листка и решение (сдача) задач листка уровня «один» дает право на получение оценки «удовлетворительно», а также ведёт к  получению заданий следующего уровня.
	\item Чтобы перейти на следующий уровень после решения задач предлагается выполнить проверочное задание по соответствующей теме соответствующего уровня. Если задание выполнено успешно, то уровень зачтен. Если нет, то ученик возвращается к изучению плохо пройденной темы вновь и получает новые задания того же уровня.
	\item Ученик может претендовать на получение следующего уровня без решения задач предыдущего. Для этого он может написать соответствующую проверочную работу, не решая задачи уровня. Если работа выполнена успешно (все верно), то ученик переводится на следующий уровень. В случае неудачи повторное выполнение проверочной работы без решения задач в этой теме для этого ученика не допускается.
	\item Через некоторое оговорённое время (обычно, когда большинство освоило уровень 2-3), задачи разбираются и переходим к следующей теме. За выполнение второго уровня ставится «хорошо», за выполнение каждого следующего уровня – «отлично».
	\end{enumerate}
\newpage
\begin{center}
	\textbf{Комментарии для учителя.}
\end{center}

Как только начинается новая тема, сначала она обсуждается в классе. Разбираются простые задачки, дается необходимый кусок теории и т.п. После этого детям выдается листок, условно называемый «Тема N. Теория. Уровень 1» В этом листке может быть приведен список задач, разобранных на занятии, короткий конспект рассказанной учителем теории, какие-то факты и задачи, которые учитель считает важным включить, но они не были рассказаны у доски. 

Работа с теоретическим листком может идти по-разному. А) можно продолжить обсуждение коллективно, предлагая решить совместно какие-то задачи из выданного листка; Б) можно включить в листок набор простеньких задач по теме и предложить выполнить их письменно; В) можно выделить время для самостоятельного решения и потом разобрать индивидуально (если хватает принимающих) или у доски для всех; Г) а можно всем выдать сразу же и листок с задачами уровня 1 со словами «А теперь вы самостоятельно решаете задачи и нам рассказываете». Однако мы считаем, что вариантом Г) злоупотреблять не стоит. В течение года это возможно, но не более 1-2 раз и не на первых темах.

После того как с теоретическим листком так или иначе покончено, дети получают листок «Тема N. Задачи. Уровень 1». Обычно в таких листках от 8 до 12 задач. Предполагается, что средний ребенок должен их решить примерно за полторы недели. Приходя на каждое занятие, ученик рассказывает решенные дома задачи принимающему до тех пор, пока задачи не кончатся. Как только это произошло, ученик получает проверочную работу из 3-4 задач (но в простых уровнях может быть и больше) на основные идеи Уровня 1 и двадцать минут на письменное решение. Очевидно, что пользоваться своими записями задач листочка, теоретическим листком и т.п. не разрешается. 
Работа считается верно выполненной ТОЛЬКО в том случае, если ВСЕ задачи решены верно и написано полное решение. Если в какой-то задаче указан только ответ без пояснений или решение с недочетами, то работа считается невыполненной. Обычно в 1 уровне таких проблем не бывает – ребенок либо решил задачу и хорошо написал, либо нет. При переходе со 2 уровня на 3 нужно быть уже более внимательным, да и задачи там уже более содержательные.

Итак, если все задачи проверочной работы решены верно, то считается, что уровень 1 успешно пройден и ученик получает листок «Тема N. Задачи. Уровень 2» и листок «Тема N. Теория. Уровень 2», если таковой имеется.

В этом месте обращаем внимание, что в силу того, что дети получают такие листки в разное время, и новая теория не разбирается, теоретический листок должен быть составлен предельно аккуратно и давать всю необходимую информацию. В частности там могут быть включены примеры решения задач. Заметим, что если в какой-то момент уже все дети получили теоретический листок уровня 2, то имеет смысл потратить часть занятия на разбор задач 1 уровня, уделяя особое внимание тем задачам, которые являются ключевыми в этой теме, и тем задачам, классическое решение которых вы хотите донести до учеников.

\textbf{Внимание!} Мы считаем, что все задачи задачного листка должны быть разобраны преподавателем тем или иным способом. Либо у доски, либо индивидуально принимающим.

Еще несколько слов про пункт 5 «правил игры». Мы специально добавили в правила такую возможность. Бывает так, что ребенок уже много решал задач такого типа, и мы знаем, что он прекрасно с ними справится. В этом случае жалко времени и хочется идти дальше. Такой ребенок обычно сразу же легко пишет проверочную работу и получает следующий листок. Бывает также, что ребенок ложно считает, что он все хорошо знает и ему простые задачки решать незачем. Мы тоже в этом случае даем ему работу. Но если знания только кажущиеся, то чаще всего такой ребенок допускает при решении задач ошибки и работу не пишет. В этом случае он теряет право получить листок 2 уровня и должен вернуться к листку 1 уровня.

Что делать, если ученик сдал все задачи листка уровня 1, но не справился с проверочной работой? В этом случае ему предлагается листок «Тема N. Задачи. Уровень 1А». В этом листке содержатся задачи, аналогичные задачам предыдущего листка, только с измененными формулировками, числами и т.п. И ученик должен снова сдать все эти задачи. Как показывает практика, при переходе с уровня 1 на уровень 2 такое встречается крайне редко, а при переходе с уровня 2 на уровень 3 – часто. Поэтому всегда стоит иметь такой дубль.

В исключительных случаях (на нашей практике такое было только один раз), если вторая попытка написать проверочную работу (она уже, конечно, другая) снова неудачна, то ребенок получает листок «Тема N. Задачи. Уровень 1Б», но прежде с ним нужно индивидуально разобрать еще раз все задачи листка А.

В зависимости от сложности темы на нее отводится от 8 до 28 уроков (от 4 до 14 занятий, если они сгруппированы парами), то есть от 2 до 7 недель. Понятно, что это условно. При необходимости можно увеличивать время до двух месяцев. Больше изучать какую-то тему нежелательно, так как она уже затирается и детям становится скучно.

Примерно один раз в три месяца мы устраиваем контрольные работы по 2-3 темам. Примеры таких работ тоже есть в книжке.  

Кроме того часть занятий отведено под игры и устные зачеты. Некоторые задачи в листочках отмечены буквой «\textit{п}». Эти задачи предназначены для письменной сдачи учениками. То есть они не принимаются устно, а требуется принести решение, записанное аккуратно и подробно дома. Эти решения проверяются, отмечаются не только верный / неверный ход решения, но и оформление, не в смысле чистописания (хотя это тоже приветствуется в разумных пределах), а в смысле строгости и аккуратности изложения.

\begin{center}
	\textbf{Расположение материала.}
\end{center}
Материалы условно разделены на несколько частей.
\begin{enumerate}
	\item Материалы, которые выдаются детям. Мы постарались сформировать их в виде уже готовом к раздаче.
	\item Комментарии для учителя и тексты для разбора в классе.
	\item Проверочные и самостоятельные работы.
	\item Решения некоторых задач и комментарии.
\end{enumerate}