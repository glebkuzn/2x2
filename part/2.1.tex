\head{Октябрь}{Теоретический Листок. Принцип Дирихле}
\epigraph{\textit{Если у тебя в двух карманах три копейки, то в одном из них меньше 2 копеек.}}{Народная мудрость}
При решении самых различных задач часто бывает полезен так называемый «принцип Дирихле», названный в честь немецкого математика Петера Густава Лежена Дирихле (1805 – 1859); по-другому этот принцип еще называют «принципом ящиков» или «принципом голубятни» . Этот принцип часто является хорошим средством при доказательстве важнейших теорем  в теории чисел, алгебре, геометрии. Возможно, вы уже слышали про этот принцип. И скорее всего, в первый раз вы услышали его формулировку в таком шутливом виде: "Пусть есть n клеток, в которых сидят не менее $n+1$ кроликов. Тогда найдется клетка, в которой сидит не меньше двух кроликов." Доказательство этого факта чрезвычайно просто. Предположим, что это не так. Т.е. в каждой клетке сидит не более одного кролика (один или ни одного). Тогда, так как клеток n, то в клетках сидит не более, чем n кроликов, что противоречит условию. Обратите внимание, что мы не можем указать, в какой именно клетке сидит больше одного кролика (даже не можем сказать, \textbf{НА} сколько больше), мы можем только утверждать, что такая клетка обязательно есть. Полученных сведений совсем мало, однако, и на основании таких, казалось бы, незначительных сведений можно делать значительные выводы.
Разберем на примерах:
\begin{thm}
	В коробке лежат шарики двух цветов. Сколько шариков достаточно наугад вынуть из коробки, чтобы среди них заведомо нашлись два одного цвета? 
\end{thm}

\begin{prf}
	Понятно, что двух шариков недостаточно. Может оказаться один черный, другой белый. Вынем три шарика. Т.к. цвета всего два, то, очевидно, что два шарика будут одного цвета. Т.е. достаточно вынуть три шарика.\footnote{Вытаскивая три шарика, мы будем уверены, что сможем выбрать из них два одного цвета, но НЕ ЗНАЕМ, КАКОГО! – черного или белого.}
\end{prf}

\begin{thm}
	В лесу растет миллион ёлок. Известно, что на каждой из них не более 400 000 иголок. Докажите, что в лесу найдутся по крайней мере три ёлки с одинаковым числом иголок.
\end{thm}

\begin{prf}
	Будем рассаживать наших «кроликов–ёлок» по «клеткам» с номерами от 0 до 400 000. В одну «клетку» будем «сажать» те ёлки, у которых число иголок равно номеру клетки. Задача будет решена, если мы докажем, что найдет «клетка», в которой «сидит» не менее трех ёлок. Всего «клеток» 400 001 штука. И если бы в каждой из них сидело не более двух «кроликов – ёлок», то всего ёлок было бы не более, чем  $2\times400001=800002$ штук. А это не так. Следовательно, найдется хотя бы одна «клетка» в которой не менее 3 «кроликов». \footnote{Мы доказали, что таких елок не менее трех, но мы не знаем точно, сколько их. Может быть, 5, может быть, 100, а может быть и весь миллион. Более того, мы не знаем, сколько конкретно, иголок на них. Может, 5, может, 400 000, а может и ни одной!}
\end{prf}

Иногда принцип рассуждений, использованный в предыдущей задаче, еще называют \textit{обобщенным принципом Дирихле}: \textit{Если в N клетках сидит не менее $(k\times N)+1$ кролик, то найдется клетка, в которой сидит не менее $k+1$ кролика.}

Заметим, что при $k =1$ обобщенный принцип Дирихле превращается в обычный принцип Дирихле.

\begin{ex}\label{u12}
	Закончите предложения:\\
	а) В пяти тарелках лежат шесть конфет. Тогда найдется тарелка, в которой лежит \dots\\
	б) В 17 чуланах живут 19 привидений. Значит найдется чулан, в котором живет \dots\\
	в) В 10 спичечных коробках лежит 21 монетка. Тогда найдется коробок, в котором лежит \dots	
\end{ex}

\begin{ex}\label{u13}	25 голубей рассадили по 7 клеткам. Укажите, какие из приведенных ниже утверждений всегда верны: \\
	а) «найдется клетка, в которой 4 голубя»;     б) «найдется клетка, в которой не менее 4 голубей»;
	в) «найдется клетка, в которой больше 4 голубей»; г) «найдется клетка, в которой 3 голубя»;\\
	д) «найдется клетка, в которой меньше 3 голубей»; е) «найдется клетка, в которой не более 3 голубей».
\end{ex}