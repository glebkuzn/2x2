\section{ Подведение итогов  и результаты.}

 Итак, подошло время подведения итогов. Поскольку это первый листок и школьники привыкают к новым правилам, к самостоятельной работе, то время, которое обычно отводится на листок -- около полутора месяцев -- примерно 12 полуторачасовых занятий. Детям заранее объявляется, когда наступает «час Х», то есть день, когда задачи будут разобраны и их решения больше принимать не будут. Лучше не устраивать разбор всех задач в самом конце, а разбирать их по ходу дела. И школьникам будет проще воспринимать, и пользы от этого больше. Например, задачи листка 1 уровня рекомендуется разобрать, когда все уже готовы писать (или уже написали) проверочную работу на 1 уровень. Часть задач второго уровня можно разбирать в процессе, когда есть такая необходимость. Если в этом случае задача рассказывается тому, кто ее не смог решить, то задачу можно заменить на похожую. Главная цель -- убедиться, что все ученики класса разобрались в решении всех задач 1 и 2 уровня.

 В уровне 3 некоторые задачи могут остаться неразобранные, если нет людей, кто над ними думал. С другой стороны, если школьники выражают желание узнать решения задач, стоит им их рассказать. 

 Мы считаем, что обязательному разбору подлежат задачи из листка 3 уровня -- \ref{1.23}, \ref{1.27}, \ref{1.31} и \ref{1.32}

 Оценки. Если эти листки решаются в школе, а не на кружке, то по правилам школы необходимым является выставление оценок. 

 Школьники, сдавшие листок 1 уровня и написавшие переводную работу на 2 уровень, но далее не продвинувшиеся, получают оценку «3» или «удовлетворительно».

 Школьники, сдавшие листок 2 уровня и написавшие переводную работу на 3 уровень, но далее не продвинувшиеся, получают оценку «4» или «хорошо».

 Школьники, сдавшие листок 2 уровня и написавшие переводную работу на 3 уровень и решившие не менее 8 задач из листка уровня 3, получают оценку «5» или «отлично». 

 Дополнительную оценку «5» можно поставить тем, кто решил все задачи листка уровня 3.

 При этом не имеет значения решил школьник задачи проверочной работы с первого или со второго раза.


\section{ Решения некоторых задач.}

\textbf{Задача \ref{1.10}}
Полный комплект костей домино выложен в цепочку. На одном конце оказалась пятерка. А что могло оказаться на другом?


\begin{prf}
	Рассмотрим множество половинок всех доминошек. Всего доминошек 28, при этом каждое число (от 0 до 6) присутствует ровно на 8 половинках. Заметим, что в выложенной цепочке все половинки кроме двух крайних разбиты на пары с одинаковыми цифрами. Это означает, что среди них любое число (от 0 до 6) встречается четное число раз. Тогда, если не рассматривать неизвестный конец, пятерка встречается на одном конце и еще на четном количестве мест, а все остальные числа встречаются на четном количестве мест. Отсюда однозначно следует, что на втором конце тоже пятерка.
\end{prf}

\textbf{Задача \ref{1.11}}
Из полного набора домино, подаренного родителями, Ваня потерял все кости с «пустышками». Сможет ли теперь кто-нибудь выложить оставшиеся кости в ряд?

\begin{prf}
	Не сможет. Предположим обратное: пусть кости разложены в ряд. Все половинки, кроме двух крайних, объединяются в пары с равными числами. На двух крайних числа могут быть как равные, так и нет (мы не можем ссылаться в этом месте на предыдущую задачу, так как часть косточек потеряна). Следовательно, все числа, за исключением двух, заведомо появляются четное число раз. Но после потери всех пустышек осталось ровно по 7 экземпляров каждой из 6 цифр 1,2,3,4,5,6. 
\end{prf}



\textbf{Задача \ref{1.23}}
Улитка ползет по плоскости с постоянной скоростью. Каждые 15 минут она поворачивает под прямым углом. Докажите, что вернуться в исходную точку она сможет только через целое число часов.
\begin{prf}
	Введем координаты. Оси направим параллельно и перпендикулярно движению улитки. А нулем обозначим ее начальное положение. Пусть за 15 минут улитка проползает 1. Тогда: \\за 15 минут улитка меняет четность одной координаты.\\ за 30 минут улитка меняет четность каждой координаты\\ за 45 минут четность хотя бы одной координаты останется измененной\\ за час четность каждой координаты не изменится\\
	Потому что изменение координаты чередуется каждые 15 минут. Значит за нецелое количество часов вернуться назад (то есть не менять четность), нельзя.
\end{prf}

\textbf{Задача \ref{1.27}}
К 17-значному числу прибавили число, записанное теми же цифрами, но в обратном порядке. Докажите, что хотя бы одна из цифр полученной суммы четна.
\begin{prf}
	Предположим противное. Посмотрим на 9 цифру: она складывается сама с собой, поэтому у суммы в 9 разряде будет четное, если нет перехода с 8 разряда. Значит есть и переход с 10 на 11 разряд (потому что участвуют те же цифры). Следовательно четность 11 и 7 цифр одинакова. \\ Поэтому у суммы в 7 разряде будет четное, если нет перехода с 6 разряда. Значит есть и переход с 6 на 7 разряд (потому что участвуют те же цифры). Следовательно четность 12 и 6 цифр одинакова.
	\\ повторяя аналогичные рассуждения приходим к тому, что четность 17 и 1 цифр одинакова.
	Но на 1 и 17 местах цифры разной четности (иначе последняя цифра суммы была бы четной). Противоречие.
\end{prf}

\textbf{Задача \ref{1.31}}
	В классе 30 учеников. Они сидят за 15-ю партами. При этом оказалось, что ровно половина всех девочек сидит с мальчиками. Докажите, что их не удастся пересадить (за те же 15 парт) так, чтобы ровно половина всех мальчиков класса сидела с девочками.
\begin{prf}
	Если половина всех девочек сидит с мальчиками, то остальная сидит с девочками, то есть по парам. Значит число девочек делится на 4. Для аналогичного условия для мальчиков требуется делимость их количества на 4. Но тогда сумма всех детей должна делиться на 4. Но 30 не делится на 4. Противоречие.
\end{prf}

\textbf{Задача \ref{1.32}}
	Сто грустных мартышек кидают друг в друга одним кокосовым орехом. Грустная мартышка, попавшая орехом в другую грустную мартышку, становится веселой и больше не грустнеет. Мартышка, в которую попали, выбывает из игры. Каких мартышек больше выбыло из игры -- веселых или грустных -- к моменту, когда в игре осталась одна мартышка?
\begin{prf}
	Веселых меньше. Чтобы попасть в веселую, она перед этим должны была попасть в грустную. Значит веселых, в которых попали, меньше или равно количеству выбитых грустных. Равенства быть не может в силу нечетности количества выбитых.
\end{prf}
% 2 уровня 12, 14, 15, 16

\textbf{Задача \ref{1.12}}
	На бирже в городе Нью-Васюки ежедневно в 10.00 проходят торги. Рано утром 1 января N-го года цены на акции фирм «Вася Inc.» и «Петя и Ко» были один и два рубля соответственно. Вечером 31 декабря того же года цены стали снова теми же. Лёша установил, что цены на акции этих фирм всегда были различны, каждый день изменялись и все время были либо один, либо два рубля. Докажите, что прошедший год был високосным.

\begin{prf}
	Каждый день четность стоимостей акций меняется. Значит такой же стоимость может стать только через четное количество дней. Только в високосном году так бывает.
\end{prf}

\textbf{Задача \ref{1.14}} Можно ли выписать в ряд по одному разу цифры от 1 до 9 так, чтобы между единицей и двойкой, двойкой и тройкой, \dots, восьмеркой и девяткой было нечетное число цифр?

\begin{prf}
	Предположим, что у нас получилось их выписать в таком порядке. Тогда единица и двойка стоят на местах одинаковой четности. Ровно как и любые другие 2 соседних по величине числа. Значит все числа стоят либо на четных, либо на нечетных местах, чего быть, конечно, не может.
\end{prf}


\textbf{Задача \ref{1.15}} 7М класс упражняется в счете. Анатолий Анатольевич написал на доске число 2011. После чего каждый ученик вышел к доске, прибавил или вычел 17 или 13 и записал получившийся результат. Когда каждый из 20 учеников вышел по одному разу, на доске оказалось написано число 2012. Анатолий Анатольевич посмотрел на доску и расстроился. Докажите, что кто-то из учеников ошибся.

\begin{prf}
	Каждая операция меняет четность числа. Через 20 операция четность не должна измениться, значит кто-то ошибся.
\end{prf}


\textbf{Задача \ref{1.16}}
	У Вини-Пуха было 2019 горшочков меда. Кристофер Робин принес или забрал 9 горшочков, что именно – Пух не помнит. На следующий день Кристофер Робин снова пришел и принес или забрал 8 горшочков, на следующий день – 7 и так далее. Наконец Кристофер Робин пришел и принес или забрал один горшочек.  а) Могло ли у Винни-Пуха на 10 день оказаться горшочков столько же, сколько и было в самом начале, то есть 2019? б) Сколько вообще горшочков меда могло быть у Вини-Пуха на 10 день, если все это время он мед не ел?

\begin{prf}
	а) за 10 дней четность количества горшочков менялась 5 раз, значит в конце должно было получиться четное количество, значит 2019 быть не могло.\\
	б) можно получить любую четную сумму от $2011-45=1966$ до $2019+45=2056$. Доказательство опустим. 
%	Пусть получена сумма $ n $, докажем, что можно получить $n+2$, если $ n $ не равно 2056. Если в последний день горшок был <<забран>> -- <<->>, то изменив забор на <<приношение>> -- <<+>> можно получить искомую сумму. В противном случае посмотрим на предыдущий день, когда Робин взял или отдал 2 горшка. 
\end{prf}

% 1 уровня -- 7, 8, 10
%
\textbf{Задача \ref{1.7}}
	На доске написано 2019 целых чисел. Всегда ли можно стереть одно из них так, чтобы сумма всех оставшихся чисел была четна?

\begin{prf}
	Да, если сумма четна, то среди чисел есть хотя бы одно четное, которое можно зачеркнуть. В противном случае есть хотя бы одно нечетное, которое тоже можно зачеркнуть.
\end{prf}


\textbf{Задача \ref{1.9}}
	Вокруг круглой поляны растут 2019 сосен. Незнайка измерил высоту каждой из них и заявил, что любые две соседние сосны отличаются по высоте ровно на метр. Знайка тут же заметил, что Незнайка врет. Кому верить?
\begin{prf}
	Высота всех деревьев, стоящих на нечетных местах одинакова. Но 1 и 2019 сосны стоят рядом, и их высоты не могут отличаться на 1. Значит Незнайка врет.
\end{prf}
