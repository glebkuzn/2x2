\newpage

\section{Решения некоторых задач.}

\begin{center}
\textbf{Уровень 1.}
\end{center} 

\textbf{Задача \ref{4.1 thm1}.} $^n$
	Докажите, что если $a_1 \del c$, $a_2 \del c$, …, $a_{n-1} \del c$, но $a_n \ndel c$, то $(a_1 + a_2 + ... + a_n) \nskdel c$.
\begin{prf}
    Предположим противное. Т. е. пусть ($a_1+a_2) \skdel c$. Тогда $a_1 + a_2 +$ ... $+ a_n = kc$. В свою очередь $a_i = b_i c$ для всех $i$ от 1 до $n-1$, т. к. по условию $a_i \del c$ для этих значений $i$. Имеем: $a_n = kc - a_1 - a_2 -...- a_{n-1} = kc - b_1 c - b_2 c -$ ... $- b_{n-1}c = c(k - b_1 - b_2 -...- b_{n-1}) = cB$, где $B$ - некоторое целое число, поскольку $k, b_1, b_2,..., b_{n-1}$ по условию целые. 
    Таким образом получаем, что $a_n$ делится на $с$ по определению делимости, что противоречит условию. Следовательно, наше предположение неверно и ($a_1 + a_2 + ... + a_n$) не делится на $c$.
\end{prf}

\textbf{Задача \ref{4.1 thm2}.} $^n$
    Группа детского сада построилась парами. Известно, что в каждой паре у одного в три раза больше конфет, чем у второго. Может ли общее число конфет быть равным 2007?
\begin{prf}
    Рассмотрим общее количество конфет в одной паре. Т. к. у одного из детей конфет в 3 раза больше, чем у другого, то количество конфет у обоих в сумме кратно 4. (если у одного $x$ конфет, то у другого $3x$ конфет и в сумме $4x$) Аналогично кратно 4 количество конфет у любой пары. Поэтому общее количество конфет у всех пар вместе также кратно 4. Но 2007 на 4 не делится. Следовательно, такое суммарное количество конфет быть не может.
\end{prf}

\textbf{Задача \ref{4.1 thm3}.} $^n$
    Петя купил общую тетрадь объёмом 96 листов и пронумеровал все её страницы по порядку числами от 1 до 192. Вася вырвал из этой тетради 25 листов и сложил все 50 чисел, которые на них написаны. Могло ли у него получиться 2006?
\begin{prf}
    Заметим, что нумеруя страницы по порядку, Петя написал на страницах одного листа последовательные числа. Поэтому на любых, в том числе и вырванных, 25-ти листах написаны 25 чётных и 25 нечётных чисел. Сумма таких 50-ти чисел всегда не четна (нечётное число нечётных чисел), поэтому 2006 получиться не могло.
\end{prf}

\begin{center}
\textbf{Уровень 2.}
\end{center}

\textbf{Задача \ref{4.2 thm1}.} $^n$
    Докажите, что при любом натуральном $n$ число $n^3 + 5n$ делится на 3.
\begin{prf}
    \par
    \textbf{\underline{Способ 1}.}
        Рассмотрим 3 случая. 
    \par
        1)~$n$ делится на 3. Но тогда поскольку и $n^3$ и $5n$ делятся в этом случае на 3, то их сумма также делится на 3. 
    \par
        2)~$n = 3k + 1$ ($n$ при делении на 3 даёт остаток 1). Тогда $n^3 = (3k +1)^3 = 27k^3 + 27k^2 + 9k + 1 = 3A + 1$, то есть $n^3$ при делении на 3 также даёт остаток 1. $5n = 5(3k + 1) = 15k + 3 + 2 = 3B + 2$, т.е. $5n$ в этом случае при делении на 3 даёт остаток 2. Отсюда следует, что $n^3 + 5n = 3A + 1 + 3B + 2 = 3(A + B + 1)$ делится на 3. 
    \par
        3)~$n = 3k + 2$ ($n$ при делении на 3 даёт остаток 2). Тогда $n^3 = (3k +2)^3 = 27k^3 + 54k^2 + 36k + 8 = 3A + 2$, то есть $n^3$ при делении на 3 даёт остаток 2. $5n = 5(3k + 2) = 15k + 9 + 1 = 3B + 1$, т. е. $5n$ в этом случае при делении на 3 даёт остаток 1. Отсюда следует, что $n^3 + 5n = 3A + 2 + 3B + 1 = 3(A + B + 1)$ делится на 3. 
    \par
        Поскольку все возможные случаи разобраны, доказательство завершено. 
    \par
    \textbf{\underline{Способ 2}.}
        Скажем, что $n^3 + 5n = n(n^2 + 5)$. Рассмотрим 2 случая.
    \par
        1)~$n$ делится на 3. Тогда и всё выражение делится на 3, т. к. включает в себя множитель, кратный трём.
    \par
        2)~$n$ не делится на 3. Тогда $n^2$ при делении на 3 может давать только остаток 1. \textit{(этот факт был разобран и доказан на уроке и есть в листке - при записи решения доказательство этого факта записывать не нужно)} Тогда $n^2+5$ делится на 3 и всё выражение делится на 3.
    \par
        Поскольку все возможные случаи разобраны, доказательство завершено. 
    \par
\end{prf}

\newpage

\textbf{Задача \ref{4.2 thm2}.} $^n$
    Число $a$ - чётное, не кратное 4. Докажите, что число $a^2$ при делении на 32 даст остаток 4.
\begin{prf}
    $a = 4k + 2$ - такой общий вид у чётных чисел, не кратных 4. Тогда $a^2 = (4k + 2)^2 = 16k^2 + 16k + 4 = 16k(k+1) + 4$. Но $k(k+1)$ - произведение двух последовательных чисел, следовательно, оно чётно. Поэтому $16k(k+1) + 4 = 32A + 4$, где $A$ - целое, что и требовалось доказать.
\end{prf}

\textbf{Задача \ref{4.2 thm3}} $^n$
    Может ли сумма квадратов двух целых чисел, не кратных 3, быть квадратом некоторого целого числа?
\begin{prf}
    Если число не кратно 3, то его квадрат при делении на 3 всегда даёт остаток 1. Тогда сумма двух таких квадратов при делении на 3 даёт остаток 2 и \textit{\underline{не может}} быть полным квадратом, потому что полный квадрат при делении на 3 даёт остаток либо 0, либо 1.
\end{prf}