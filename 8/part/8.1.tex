\head{Сентябрь}{Листок 8. Логика. Уровень 0.}

\epigraph{\textit{– Разве это ложь? – сказала Королева. – Слыхала я такую
ложь, рядом с которой эта правдива, как толковый словарь!}}{\textit{Л. Кэрролл "Алиса в Зазеркалье"}}

Безусловно, вы уже сталкивались с задачами, в которых нужно рассматривать случай, когда кто-то лжёт. Соответственно встаёт вопрос об истинности или ложности некоторых утверждений. Позже мы разберём это подробнее, а пока то, что есть «правда», а что «ложь», мы будем считать интуитивно понятным. Для начала несколько простых упражнений.
\\
Напомним, что «рыцари» всегда говорят правду, а «лжецы» всегда лгут.

\begin{ex}
    Известно, что в ящике лежит 17 шариков. Какие из приведенных ниже утверждений являются всегда истинными, а какие – всегда ложными
    \begin{enumerate}[label=\asbuk*), ref=\asbuk*]
        \item В ящике не менее 17 шариков.
        \item В ящике лежит 17 шариков или 15 шариков.
        \item В ящике найдётся 15 шариков.
        \item В ящике есть шарики разных цветов.
        \item В ящике найдётся не более 17 одноцветных шариков.
        \item В ящике не менее двух шариков одного размера.
        \item Из ящика нельзя вытащить 20 шариков одного цвета.
        \item Шарики в ящике можно сложить по парам.
    \end{enumerate}
\end{ex}

\begin{ex}
    После победы над Змеем Горынычем три богатыря заявили:
    \par
    \underline{Илья Муромец}: «Змея убил Добрыня Никитич».
    \par
    \underline{Добрыня Никитич}: «Змея убил Алёша Попович».
    \par
    \underline{Алёша Попович}: «Змея убил я».
    \par
    Кто убил Змея и почему именно он, если только один из них сказал правду?
\end{ex}

\begin{ex}
    В комнате находятся рыцарь и лжец. Кто из них мог сказать фразу: «Мы оба лжецы»?
\end{ex}

\begin{ex}
    На острове рыцарей и лжецов житель $A$ в присутствии другого жителя $B$ говорит: «По крайней мере один из нас – лжец». Кто такие $A$ и $B$?
\end{ex}

\begin{ex}
    В комнате 2012 жителей острова рыцарей и лжецов. Каждый из них заявил: «Кроме меня в комнате все лжецы!». Сколько рыцарей в комнате?
\end{ex}

\begin{ex}
    В комнате 2012 жителей острова рыцарей и лжецов. Одного из них зовут Ваня. Они встали в круг, и Миша – правый сосед Вани – сказал: «Ваня, ты лжец!». Тогда правый сосед Миши сказал: «Ты не прав!», и так далее: каждый (кроме Вани) по кругу высказал своему соседу, что он неправ. Сколько в комнате лжецов?
\end{ex}

\begin{ex}
    А если в предыдущей задаче каждый (кроме Вани) сказал «Ты прав!», то сколько тогда было бы в комнате лжецов?
\end{ex}

\begin{ex}
    У Снусмумрика украли флейту. Известно, что те, кто крадут флейты, всегда лгут. «Я знаю, кто украл флейту!», – заявил Мумми-тролль. Виновен ли Мумми-тролль?
\end{ex}

\begin{ex}
    Среди трёх человек $A$, $B$ и $C$ один лжец, один рыцарь, а третий – нормальный человек, который может говорить и правду, и ложь. 
    $A$ говорит: «Я нормальный человек».
    $B$ говорит: «$A$ и $C$ иногда говорят правду». 
    $C$ говорит: «$B$ – нормальный человек».
    Кто из них кто?
\end{ex}

\begin{ex}
     В конференции участвовало 100 человек – химиков и алхимиков. Каждому был задан вопрос: «Если не считать Вас, то кого больше среди остальных участников – химиков или алхимиков?». Когда опросили 51 участника, и все ответили, что алхимиков больше, опрос прервался. Алхимики всегда лгут, а химики говорят правду. Сколько химиков среди участников?
\end{ex}

\begin{ex}
    В комнате четыре человека – жители острова рыцарей и лжецов каждый из них сделал заявление.
    \par
    Первый: «среди нас не более одного лжеца».
    \par
    Второй: «среди нас не более двух лжецов».
    \par
    Третий: «среди нас не более трех лжецов».
    \par
    Четвёртый: «среди нас не более четырех лжецов».
    \\
    Сколько рыцарей в комнате?
\end{ex}

\begin{ex}
    У Карабаса-Барабаса украли пять золотых. Карабас подозревает в краже лису Алису, кота Базилио, Дуремара и Буратино, так как неопровержимыми уликами установлено, что 
    \par
    кто-то из них обязательно виновен; 
    \par
    никто больше не мог это сделать;
    \par
    Алиса всегда действует заодно с Базилио;
    \par
    если Дуремар виновен, то у него было ровно 2 соучастника;
    \par
    если Буратино виновен, то у него был ровно 1 соучастник.
    \\
    Нужно определить, виновен ли Базилио.
\end{ex}

\section{Наконец-то, задачи!}

В следующих задачах не требуется ничего, кроме здравых рассуждений. Напоминаем, что задачи, отмеченные значком \textit{п}, являются письменными, и в устном виде приниматься не будут. Отмеченные звёздочками не являются обязательными (это не значит, что они сложнее).

\begin{thm} $^n$
    Мама пришла с работы домой и обнаружила, что коробка с конфетами пуста. На вопрос «Кто съел конфеты?» её дочери Вета, Рая и Соня ответили так:
    \par
    Вета: «Соня не ела последнюю конфету» 
    \par
    Рая: «Соня и Вета обе ели конфеты»
    \par 
    Соня: «Рая и Вета обе не ели конфеты».
    \\ 
    Впоследствии оказалось, что все сказали неправду. Кто съел конфеты?
\end{thm}

\begin{thm}
    На острове невезения живут 2012 человек. Некоторые из них всегда лгут, а остальные всегда говорят правду. Каждому жителю острова не везёт только в один из трёх дней: понедельник, среду или пятницу. Однажды каждому жителю острова задали три вопроса:
    \par
    1. «Вам не везёт в понедельник?»
    \par 
    2. «Вам не везёт в среду?»
    \par
    3. «Вам не везёт в пятницу?»
    \\
    На первый вопрос ответили «да» 999 человек, на второй – 1000, на третий – 1001. Сколько лжецов на острове?
\end{thm}

\begin{thm}
    В комнате сидело 2012 жителей острова рыцарей и лжецов. В какой-то момент один человек обиделся и ушёл. Один из оставшихся, поглядев в след, заметил: «Ушедший – лжец!» После чего встал и тоже вышел. Второй сказал: «Оба ушедшие – лжецы» и тоже ушёл. Далее каждый из оставшихся уходил, говоря: «Все ушедшие – лжецы». Пока последний оставшийся в комнате печально не констатировал: «Да, все ушедшие – лжецы».
    \\
    Определите, сколько в комнате было лжецов первоначально. (Лжецы всегда лгут, рыцари всегда говорят правду)
\end{thm}

\begin{thm}
    На планете «Куб» (имеющей форму куба) каждой гранью владеет рыцарь или лжец. Каждый из них утверждает, что среди его соседей лжецов больше, чем рыцарей. Сколько рыцарей и сколько лжецов владеют гранями планеты?
\end{thm}

\begin{thm}
    Однажды каждый житель планеты «Куб» из предыдущей задачи (возможно, теперь они владеют гранями планеты как--то по--другому) сделал заявление: «Среди моих соседей лжецов больше, чем рыцарей». Можно ли поменять местами двух человек, чтобы каждый из них мог сказать «Среди моих соседей рыцарей больше, чем лжецов»?
\end{thm}

\newpage

\begin{thm}
    Министры иностранных дел Ассирии, Аримака и Такито обсудили за закрытыми дверями проекты соглашения о полном разоружении, представленные каждой из стран. Отвечая затем на вопрос журналистов: «Чей именно проект был принят?», министры дали такие ответы:
    \par
    Ассирия: «Проект не наш. Проект не Аримаки»;
    \par
    Аримака: «Проект не Ассирии. Проект Такито»;
    \par
    Такито: «Проект не наш. Проект Ассирии».
    \\
    Один из них (самый откровенный) оба раза говорил правду; второй (самый скрытный) оба раза говорил неправду, третий (осторожный) один раз сказал правду, а другой раз – неправду. Определите, чей проект был принят.
\end{thm}

\begin{thm}
    а) Обязательно ли является ли старейший шахматист среди музыкантов старейшим музыкантом среди шахматистов?
    \par
    б) Обязательно ли является ли лучший шахматист среди музыкантов лучшим музыкантом среди шахматистов?\footnotemark
\end{thm}\footnotetext{Подумайте сначала, чем отличается шахматист среди музыкантов от музыканта среди шахматистов.}