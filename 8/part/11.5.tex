\head{Апрель}{Листок 11. Делимость целых чисел -- 2. Уровень 2}

\section{В задачах этого листка речь идёт только о целых числах.}

\begin{figure}[h]


    \begin{minipage}{0.25\linewidth}
        \includegraphics[width=0.95\columnwidth]{img/11.5 sbpch.png}
    \end{minipage}
    \hfill
    \begin{minipage}{0.73\linewidth}
\begin{thm}
    Пусть $p$ -- простое. Докажите, что тогда либо $a$ делится на $p$, либо НОД($a$, $p$) = 1.
\end{thm}

\begin{thm}
    \textbf{а)} Докажите, что если четырехзначное число $p$ не делится ни на одно простое число от 2 до 97, то $p$ -- простое. \textbf{б)} Докажите, что каждое составное число $a$ имеет простой делитель $p$ такой, что $p_2 \leq a$.
\end{thm}

\begin{thm}
    Докажите, что множество простых чисел бесконечно.
\end{thm}

\begin{thm}
    Известно, что $p$, $p$ + 10 и $p$ + 14 -- простые числа. Найдите $p$.
\end{thm}

        
    \end{minipage}

\end{figure}
    
\begin{thm} $^n$
    Известно, что $p$, $4p_2$ + 1 и $6p_2$ + 1 -- простые числа. Найдите $p$.
\end{thm}

\begin{thm}
    Известно, что $p$ и $p^2$ + 2 -- простые числа. Докажите, что $p^3$ + 2 также простое число.
\end{thm}

\begin{thm}
    Докажите, что если $(n - 1)! + 1$ делится на $n$, то $n$ -- простое число.
\end{thm}

\begin{thm}
    Докажите, что существует такое натуральное $n$, что числа $n + 1$, $n + 2$, ..., $n + 2013$ -- составные.
\end{thm}

\begin{thm}
    Пользуясь идеями двух предыдущих задач, докажите, что для любого $n$ существует $n$ подряд идущих составных чисел.
\end{thm}

\begin{thm}
    Докажите, что среди членов каждой из арифметических прогрессий:
    \par а) 3, 7, 11, 15, 19, ... \hfill б) 5, 11, 17, 23, 29, ... \hfill в) 11, 21, 31, 41, 51, ...
    \par имеется бесконечно много простых чисел.
\end{thm}