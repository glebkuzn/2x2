\head{Сентябрь}{Листок 8. Логика. Математический бой.}

\begin{enumerate}

    \item 
    Каждый житель острова Невезения — либо рыцарь, который всегда говорит правду, либо лжец, который всегда лжёт, причём лжецов на острове ровно 33. Однажды каждый житель острова заявил: «Среди всех жителей острова, не считая меня, не меньше трети лжецов». Сколько жителей может быть на острове? Перечислите все возможности и докажите, что других возможностей нет.
    
    \item 
    На собрании лжецов и рыцарей путешественник пытается определить самого старшего. Ему известно, что среди присутствующих лжецов и рыцарей поровну, а также, что возрасты всех различны. Ему разрешается выбрать любую группу людей и спросить любого из присутствующих, кто в этой группе самый старший. Докажите, что путешественник не сможет гарантированно определить самого старшего, сколько бы вопросов он ни задавал.
    
    \item 
    Рыцари и лжецы встали в хоровод, при этом некоторые из них знакомы, а некоторые -- нет. Каждый сказал своему левому соседу: «Я знаю, кто ты, и я знаю, что ты лжец». Докажите, что лжецов в круге не меньше половины.
    
    \item 
    Каждый островитянин произнёс две фразы: «Среди моих знакомых островитян более двух лжецов» и «Среди моих знакомых островитян более трёх рыцарей». Докажите, что лжецов на острове не меньше, чем рыцарей

    \item 
    На острове Невезения ровно 1 житель сказал: «Есть лжец выше меня», 
    \\ ровно 2 жителя сказали: «Есть хотя бы двое лжецов выше меня», 
    \\ ... , 
    \\ ровно 20 жителей сказали: «Есть хотя бы 20 лжецов выше меня». 
    \par Известно, что все островитяне — разного роста, и каждый житель высказался ровно один раз. Сколько лжецов живёт на острове? Напомним, что рыцари всегда говорят правду, а лжецы всегда лгут.

    \item 
    Среди 300 человек есть 100 Петь, 100 Гош и 100 Мить. После того, как каждому задали вопрос, как его зовут, получилось 100 ответов «Петя», 100 ответов «Гоша» и 100 ответов «Митя». Известно, что ровно 80 Петь и ровно половина Гош всегда лгут, а остальные Пети и Гоши всегда говорят правду. Какое наибольшее число Мить могут быть кристально честными?

    \item 
    На острове живут рыцари, которые всегда говорят правду и лжецы, которые всегда лгут. Встретились три островитянина: Михаил Михайлович, Сергей Сергеевич, и Анатолий Анатольевич. Михаил сказал: «Мы все лжецы». Сергей на это ему ответил: «Нет, только ты». Кто есть Анатолий  — рыцарь, или лжец?

    \item 
    На острове появилось новое сообщество -- хитрецы, которые иногда говорят правду, а иногда лгут. За круглым столом сидят представители рыцарей, лжецов, и хитрецов. Каждый из сидящих произнёс две фразы:
    \par 1) «Слева от меня сидит лжец.» 
    \par 2) «Справа от меня сидит хитрец.»
    \\ Докажите, что за этим столом рыцарей сидит столько же, сколько и лжецов.
    
    \item 
    Дорогу разделили на 2011 частей (не обязательно одинаковой длины). Яна знает длину дороги. Ей разрешено спросить, чему равно расстояние между серединами любых двух частей, при этом количество вопросов не ограничено. Длину каких частей она сможет узнать?

    \item 
    \textit{Лесногородская головоломка}. 
    \par Все жители санатория «Лесной Городок» делятся на 4 типа:
    \\ 1) Титаны в здравом уме; 
    \\ 2) Титаны, лишившиеся рассудка;
    \\ 3) Друиды, находящиеся в здравом уме
    \\ 4) Друиды, лишённые рассудка. 
    \\ Титаны в здравом уме всегда говорят истину (их утверждения правильны и сами титаны честны). Титаны, лишённые рассудка, всегда лгут (в силу собственных заблуждений, но не умышленно!). Друиды в здравом уме тоже всегда лгут (в силу своей природы, а не по заблуждению). 
    \\ Друиды, лишённые рассудка, всегда говорят правду (они убеждены в своих ложных утверждениях, но при этом умышленно лгут).
    \\ Однажды три Титана делились впечатлениями о прогулках по санаторию, совершенными в разное время. Первый Титан сказал:
    \\ -- Я встретил одного человека по имени Лёша и спросил его, является ли он Титаном в здравом уме. Он мне ответил вполне определённо («да» или «нет»), но я так и не понял, к какому типу он принадлежит.
    \\ -- Я тоже встретил этого Лёшу, -- сказал второй Титан. -- Я спросил у него, является ли он Друидом, лишившимся рассудка, и он ответил определённо («да» или «нет»), однако я не понял, кем он был в действительности.
    \\ -- Я тоже встретил этого Лёшу! -- воскликнул третий Титан. -- Я спросил у него, является ли он Друидом в здравом уме, и он также ответил определённо («да» или «нет»), однако я всё равно не понял, кто он!
    \\ Кто же такой -- Лёша?
\end{enumerate}