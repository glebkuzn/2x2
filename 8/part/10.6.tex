\head{Февраль}{Листок 10. Графы. Уровень 4.}

\section{Эйлеровы и гамильтоновы обходы. Ориентированные графы.}

\begin{thm}
    Доказать, что в графе, соответствующем задаче про коней из листка с теорией есть гамильтонов путь.
\end{thm}

\begin{ex}
    Изобразите на додекаэдре гамильтонов цикл.
\end{ex}

\begin{thm}
    В большой книге предсказаний Глеба Лобы написано:
    \begin{enumerate}[itemsep=0.05em]
        \item Если сегодня дождь, то завтра будет солнце.
        \item Если сегодня снег, то завтра дождь.
        \item Если сегодня холод, то завтра будет ветер.
        \item Если сегодня солнце, то завтра будет тепло
        \item Если сегодня тепло, то завтра будет холодно.
        \item Если сегодня холодно, то завтра будет пасмурно.
        \item Если сегодня ветер, то завтра будет снег.
        \item Если сегодня пасмурно, то завтра будет дождь.
    \\ Оказалось, что в январе все предсказания сбылись. 1 января были ветер и солнце. Какая погода была 5 января?
    \end{enumerate}
\end{thm}

\begin{thm}
    В однокруговом шахматном турнире один шахматист заболел и не доиграл свои партии. Всего в турнире было проведено 24 встречи. Сколько шахматистов участвовало в турнире, и сколько партий сыграл выбывший участник?
\end{thm}

\begin{thm}
    Докажите, что на рёбрах любого связного графа можно так расставить стрелки, что найдётся вершина, из которой можно было бы добраться по стрелкам в любую другую.
\end{thm}

\begin{thm}
    Сумасшедший король хочет ввести на дорогах своего королевства одностороннее движение так, чтобы, выехав из одного города, уже будет нельзя в него вернуться. Удастся ли ему осуществить свою затею, если в его государстве любые два города соединены дорогой?
\end{thm}

\begin{thm}
    В некоторой стране каждый город соединён с каждым дорогой с односторонним движением. Докажите, что найдётся город, из которого можно попасть в любой другой.
\end{thm}

\begin{thm}
    В одном государстве 100 городов и каждый соединён с каждым дорогой с односторонним движением. Докажите, что можно поменять направление движения на одной дороге так, что из любого города можно будет доехать до любого другого.
\end{thm}

\begin{thm}
    Ученики одного класса сыграли между собой круговой турнир по настольному теннису. Будем говорить, что игрок А сильнее игрока В, если либо А выиграл у В, либо существует игрок С, у которого А выиграл, а В ему проиграл.
    \\ а) Докажите, что есть игрок, которых сильнее всех.
    \\ б) Докажите, что игрок, выигравший турнир, сильнее всех. (\underline{\textit{Замечание:}} в теннисе ничьих не бывает)
\end{thm}

% \begin{thm}
%     Докажите, что шахматный конь может обойти доску $4 \times n$, побывав в каждой клетке по одному разу. Может ли он при этом вернуться в исходную клетку?
% \end{thm}

\begin{thm}
    При дворе короля Артура собрались $2n$ рыцарей, причём каждый из них имеет среди присутствующих не более $n - 1$ врага. Доказать, что Мерлин может так рассадить рыцарей за круглым столом, что ни один из них не будет сидеть рядом с врагом.
\end{thm}

\begin{thm}
    Докажите, что гамильтонов граф двусвязен.
\end{thm}

\begin{thm} $^*$
    Докажите теорему Поша.
\end{thm}

\begin{thm} $^*$
    Выведите теоремы Дирака и Оре из теоремы Поша.
\end{thm}

\begin{thm}
    Докажите, что в двусвязном негамильтоновом графе содержится тэта--граф в качестве подграфа.
\end{thm}