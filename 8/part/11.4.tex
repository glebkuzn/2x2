\head{Апрель}{Листок 11. Делимость целых чисел -- 2. Уровень 1}

\section{В задачах этого листка речь идёт только о целых числах.}

\begin{thm}
     Выберите из следующих чисел все пары взаимно простых чисел: 10, 12, 17, 25, 44, 77, 68, 121.
\end{thm}

\begin{thm}
    Найдите: \textbf{а)} НОД(105; 77), НОК (105, 77); \textbf{б)} НОД(100!, 102!); НОК(100! + 101!; 101! + 102!)
\end{thm}

\begin{thm}
    При каких натуральных $n$ будут взаимно простыми числа: \textbf{а)} $10n + 1$ и $3n + 2$ \textbf{б)} $(n + 1)(n + 2)$ и $2n$?
\end{thm}

\begin{thm}
    При каких натуральных $n$ \textbf{а)} число $n + 13$ делится на $n + 3$; \textbf{б)} число $n^3 + 27$ делится на $7 - n$?
\end{thm}

\begin{thm}
    Верно ли, что, если НОД($a$, $b$) = НОД($b$, $c$) = $d$, то и НОД($a$, $c$) = $d$?
\end{thm}

\begin{thm}
    Докажите, что если НОД($a$, $b$) = $d$ и $a$ = $a_1d$, $b$ = $b_1d$, то числа $a_1$, $b_1$ взаимно простые.
\end{thm}

\begin{thm} $^n$
    Докажите, что любой общий делитель чисел $a$ и $b$ является делителем НОД($a$, $b$).
\end{thm}

\begin{thm}
    Докажите, что если НОК($a$, $b$) = $k$ и $m$ -- натуральное число, то НОК($am$, $bm$) = $km$.
\end{thm}

\begin{thm} $^n$
    Докажите, что если НОК($a$, $b$) = $k$ и НОД($a$, $b$) = $d$, то НОК $\left( \dfrac{a}{d}, \dfrac{b}{d} \right) = \dfrac{k}{d}$.
\end{thm}

\begin{thm}
    Докажите теорему \ref{11.0 thrm1} из теоретического листка: для любых натуральных чисел $a$ и $b$ справедливо $ab$ = НОД($a$, $b$) $\times$ НОК($a$, $b$).
\end{thm}

\begin{thm}
    Докажите, что если $b | (ca)$ и НОД($a$, $b$) = 1, то $b | c$.
\end{thm}

\begin{thrm}
    Пусть $ax = by$ и НОД($a$, $b$) = 1. Докажите, что тогда найдётся такое число $t$, что $x = bt$, $y = at$.
\end{thrm}