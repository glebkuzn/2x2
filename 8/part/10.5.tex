\head{Январь}{Листок 10. Графы. Уровень 3.}

\section{Деревья. Изоморфные графы.}

\begin{thm}
    Докажите, что граф является деревом тогда и только тогда, когда каждые две его вершины соединены ровно одним путём с различными ребрами.
\end{thm}

\begin{thm}
    В некотором графе все вершины имеют степень три. Докажите, что в нём есть цикл.
\end{thm}

\begin{thm}
    а) Докажите, что в дереве с $n \geq 2$ вершинами найдутся хотя бы две висячие вершины. 
    \par б) Можете ли вы привести пример графа со 101 вершиной, из которых 100 висячих?
\end{thm}

\begin{thm}
    Ребра дерева окрашены в два цвета. Если в какой-то вершине сходятся ребра одного цвета, то можно их все перекрасить в другой цвет. Можно ли всё дерево сделать одноцветным?
\end{thm}

\begin{thm}
    Будем красить в два цвета не рёбра, а вершины графа. Можно ли любое дерево раскрасить так, что любое ребро будет соединять вершины разных цветов?
\end{thm}

\begin{thm}
    Волейбольная сетка имеет вид прямоугольника размером 60х600 клеток. Хулиган Лёша хочет разрезать как можно больше верёвочек так, чтобы сетка не распалась на отдельные куски. Сколько веревочек ему удастся разрезать?
\end{thm}

\begin{thm}
    На планете Абра-Кадабра 100 государств, некоторые из которых соединены авиалиниями. Известно, что из любого государства можно попасть в любое другое (возможно, с пересадками). Докажите, что можно совершить кругосветное путешествие (побывать в каждой стране), сделав не более 196 перелётов.
\end{thm}

\begin{thm}
    Докажите, что в любом связном графе можно удалить вершину вместе со всеми выходящими из неё ребрами так, чтобы он остался связным.
\end{thm}

\begin{thm}
    В группе из нескольких человек некоторые люди знакомы друг с другом, а некоторые нет. Каждый вечер один из них устраивает ужин для всех своих знакомых, на котором знакомит их друг с другом. После того, как каждый человек устроил хотя бы по одному ужину, оказалось, что какие--то два человека вcё ещё не знакомы. Докажите, что они не познакомятся и на следующем ужине.
\end{thm}

\begin{dfn}
    Два графа называются \textit{изоморфными}, если у них поровну вершин, и если  вершины каждого графа можно занумеровать числами от 1 до $n$ так, что вершины $k$ и $l$ соединены ребром в одном графе тогда и только тогда, когда соединены ребром вершины $k$ и $l$ в другом графе.
\end{dfn}

\begin{thm}
    Верно ли, что два графа изоморфны, если
        \begin{enumerate}[noitemsep, label=\asbuk*), ref=\asbuk*]
            \item у них по 10 вершин, степень каждой из которых равна 9?
            \item у них по 8 вершин, степень каждой из которых равна 3, и нет петель?
            \item они связны, без циклов и петель, содержат 5 вершин и 4 ребра?
        \end{enumerate}
\end{thm}

\begin{thm}
    В связном графе степени четырёх вершин равны 3, а степени остальных равны 4. Докажите, что нельзя удалить одно ребро так, чтобы этот граф распался на две изоморфные компоненты связности.
\end{thm}

\begin{thm} $^*$
    Маша нарисовала на доске 7 графов, каждый из которых является деревом с шестью вершинами. Докажите, что среди них найдутся, по крайней мере, два изоморфных.
\end{thm}
