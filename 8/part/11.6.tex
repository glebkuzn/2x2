\head{Апрель}{Листок 12. Дополнительный. Системы счисления.}

«Из подъезда вышел человек лет около 49; пройдя по улице метров 196, он зашел в магазин, купил там две семёрки яиц и пошёл дальше...». Такое описание звучит несколько странно, не правда ли? Обычно, приблизительно оценивая какую--либо величину, мы пользуемся круглыми числами: «метров 200», «лет 50» и т.п. Говоря о круглых числах, мы обычно не отдаём себе отчёта в том, что деление чисел на круглые и некруглые по существу условно, и одно и то же число может быть круглым или некруглым в зависимости от того, какой системой счисления мы пользуемся.

Прежде разберёмся, что представляет собой наша десятичная система счисления, которой мы обычно пользуемся. Например, запись 2345 означает, что данное число содержит 5 единиц, 4 десятка, 3 сотни и 2 тысячи, т.е. 2345 -- сокращённое обозначение выражения

\begin{equation}
    2 \times 10^3 + 3 \times 10^2 + 4 \times 10^1 + 5 \times 10^0.
\end{equation}

Мы получили разложение по степеням числа 10. Как в первом классе, объясняя нам, что такое десятки, учительница предлагала связать в пучок 10 палочек и назвать этот пучок \textit{десятком}, связав десять таких пучков в больший пучок назвать последний \textit{сотней} и т.д. И запись нашего числа означает всего лишь сколько каких “пучков“ надо взять, чтобы получить рассматриваемое число. Однако с таким же успехом можно представить любое число в виде комбинации степеней не числа 10, а любого другого натурального числа (кроме 1), например, семи. В этой системе, называемой \textit{семеричной системой счисления} или \textit{системой счисления с основанием 7}, мы вели бы счёт от 0 до 6 обычным образом, а число 7 приняли бы за единицу следующего разряда. (Тогда бы в первом классе мы связывали бы пучки не по 10 палочек, а по семь.) Т.е. само число 7 в нашей семеричной системе счисления было бы обозначено символом 10. Чтобы не путать обозначения в разных системах счисления принято нижним индексом обозначать, в какой системе счисления мы работаем: 710 = 107.

\begin{thm}
    Верно ли, что для любого натурального числа $p > 1$ справедливо равенство $p_{10} = 10_р$?
\end{thm}

\begin{dfn} \label{11.6 dfn1}
    $\left( \overline{a_ka_{k-1}...a_1a_0} \right)_p = a_k \times p^k + a_{k - 1} \times p^{k - 1} + ... + a_1 \times p + a_0$, \hfill (*)
    \par где натуральные числа $a_i < p$.   
\end{dfn}

\noindent Заметим, что в $p$--ичной системе счисления круглыми будут совсем не те числа, которые были круглыми в десятичной системе. Например, в приведённом в самом начале все используемые в рассказе числа являются круглыми в семеричной системе счисления.

\begin{ex} $^*$
    Запишите числа в рассказе в семеричной системе счисления.
\end{ex} 

\noindent Системы счисления, о которых мы говорим, строятся по одному общему принципу. Выбирается некоторое число $p$ -- \textit{основание системы счисления}, и каждое число $N$ представляется в виде комбинации его степеней с коэффициентами, принимающими значения от 0 до $p - 1$ (см. Определение \ref{11.6 dfn1}). В такой записи значение каждой цифры зависит от того места, которое она занимает. Системы счисления, построенные таким образом, называются \textit{позиционными}.

Существуют также и \textit{непозиционные} системы счисления. Общеизвестный пример такой системы -- римские цифры. В этой системе имеется набор основных символов, и каждое число представляется как комбинация этих символов. Например, число 288 запишется в этой системе как CCLXXXVIII. В этой системе смысл каждого символа не зависит от места, на котором он стоит. Цифра Х, участвуя три раза, обозначает одну и ту же величину -- 10 единиц. А, например, в десятичной системе счисления в числе 222 каждая из двоек имеет разный смысл. Далее мы будем рассматривать только позиционные системы счисления.

Для чисел, записанных в десятичной системе счисления, мы пользуемся правилами сложения и умножения «столбиком», для деления -- «уголком». Те же правила действуют и для чисел, записанных в других системах счисления.

\begin{figure}
    \begin{minipage}{0.2\linewidth}
        \begin{tabular}{ |c|c|c| } 
         \hline
         сл. & 1 & 2 \\ 
         \hline
         1 & 2 & 10 \\ 
         \hline
         2 & 10 & 11 \\ 
         \hline
        \end{tabular}
    \end{minipage}
    \hfill
    \begin{minipage}{0.2\linewidth}
        \begin{tabular}{ |c|c|c| } 
         \hline
         ум. & 1 & 2 \\ 
         \hline
         1 & 1 & 2 \\ 
         \hline
         2 & 2 & 11 \\ 
         \hline
        \end{tabular}
    \end{minipage}
    \hfill
    \begin{minipage}{0.5\linewidth}     
        \textbf{Пример}: таблицы сложения и умножения
        \par для троичной системы счисления.
    \end{minipage}
\end{figure}

\begin{thm}
    Составьте таблицы сложения и умножения для семеричной и восьмеричной систем счисления.
\end{thm}


\begin{figure}
    \begin{minipage}{0.7\linewidth}       
        \begin{thm}
            После урока спец. математики в 8 классе на доске сохранилась 
            полустёртая надпись. Выясните, в какой системе счисления это было записано.
        \end{thm}
    \end{minipage}
    \hfill
    \begin{minipage}{0.21\linewidth}
        \begin{tabular}{ rrrrrr } 
          &2 & 3 & * & 5 & * \\
         + & 1 & * & 6 & 4 & 2 \\
         \hline
          & 4 & 2 & 4 & 2 & 3
         \end{tabular}
    \end{minipage}
\end{figure}

\begin{thm}
    Если цифру 4, которой оканчивается семеричная запись четырёхзначного числа, перенести в начало, то число увеличится в 3 раза. Найдите это число.
\end{thm}

\begin{thm}
    Если цифру 4, которой оканчивается семеричная запись четырехзначного числа, перенести в начало, то число увеличится в 3 раза. Найдите это число.
\end{thm}
 
\begin{thm} $^*$
    Существуют ли натуральные числа, которые увеличиваются в 4 раза после переноса в конец первой цифры в их 8--ричной записи?
\end{thm}

\begin{thm} $^*$
    В какой р--ичной системе счисления существуют неоднозначные натуральные числа, равные произведению своих цифр?
\end{thm}

\begin{thm}
    Верно ли, что $2(p - 1)$ и $(p - 1)^2$ в $р$--ичной системе счисления записываются одними и теми же цифрами, но в обратном порядке?
\end{thm}

\section{Перевод чисел из одной системы счисления в другую.}

Представить какое либо число A в $p$--ичной системе счисления -- это значит представить его в виде (*).
Следовательно, надо найти коэффициенты $a_0, a_1, ...$ Приведём алгоритм этого поиска на примере перевода числа 328710 в семеричную систему счисления.

\begin{center}
\begin{tabular}{|m{7.5cm}|m{9cm}|}
        \hline      
    \textbf{\textit{Шаг 1.}} Разделим заданное число A с остатком на $p$. Очевидно, что остаток при этом будет равен $a_0$. & \textbf{\textit{Шаг 1.}} А = 3287 = $469 \times 7$ + 4 (символ десятичной системы опускаем) \hfill $p = 7$ \hfill $a_0 = 4$ \\
        \hline
    \textbf{\textit{Шаг 2.}} Разделим полученное частное снова на $p$
    Остаток при этом будет равен $a_1$. & \textbf{\textit{Шаг 2}}. 469 = $67 \times 7$ + 0 \hfill  $a_1 = 0$ \\
        \hline
    \textbf{\textit{и так далее, пока частное не будет меньше р}} & \textbf{\textit{Шаг 3.}} 67 = $9 \times 7$ + 4 \hfill $a_2 = 4$ \\
        \hline
    \textbf{\textit{Последний шаг.}} Это последнее частное является цифрой самого старшего разряда в искомом разложении. & \textbf{\textit{Шаг 4.}} 9 = $1 \times 7$ + 2 \hfill  $a_3 = 2$ \par
    \hfill $a_4 = 1$ \\
    \hline
     & \textbf{\textit{Окончательный ответ:}} $3287_{10} = 12404_7$ \\
     \hline
\end{tabular}
\end{center}

\begin{ques}
    Если исходное число записано не в десятичной, а в некоторой другой системе счисления, то будет ли верна приведенная выше схема перевода числа в $p$--ичную систему? И как при этом надо действовать?
\end{ques}

\begin{ex}
    Переведите в восьмеричную и девятеричную систему счисления числа $19998_{10}$ и $16600_7$.
\end{ex}

\begin{ex}
    Запишите в троичной системе счисления числа $3_{10}, 8_{10}$ и $26_{10}$.
\end{ex}

\begin{thm}
    Придумайте признаки делимости на 2, 3, 4, 5 и 7 для чисел, записанных в троичной системе счисления.
\end{thm}

\begin{thm}
    Докажите что $\overline{cdcdcdcdcd_{11}}$ не делится на $\overline{aabb_{11}}$ ни при каких $a, b, c, d$.
\end{thm}

\begin{thm}
    Существует ли такое натуральное число (в десятичной записи), которое при делении на сумму своих цифр даёт 2011 как в частном, так и в остаткe?
\end{thm}

\newpage

\section{Применение различных систем счисления.}

\begin{thm}
а) Предположим, что имеются чашечные весы и гири в 1г, 3г, 9г, 27 г и т.д. (по одной штуке каждого веса). Можно ли с помощью такого набора гирь взвесить любой груз с точностью до 1 грамма?  
\\
б) Какие вы можете предложить наборы гирь, подходящие для описанной выше цели?  
\end{thm}

\begin{thm}
а) Имеется 10 мешков с монетами. В 9 мешках монеты -- настоящие, а в одном -- фальшивые. Известно, что настоящая монета весит 10г, а фальшивая -- 11г. Одним взвешиванием на точных весах со стрелкой определите, в каком мешке фальшивые монеты. 
\\
б) \textbf{*}  Можете ли вы это сделать, если неизвестен точный вес монет, а известно только, что фальшивые монеты на 1г тяжелее настоящих?
\end{thm}

\begin{thm}
а) На этот раз среди десяти мешков с монетами, возможно, есть несколько мешков с фальшивыми монетами (а, возможно, ни одного!). Можно ли одним взвешиванием на точных весах со стрелкой определить, в каких именно мешках фальшивые монеты? Вес фальшивой монеты -- 11г, настоящей -- 10г.
\\
б) \textbf{*} Можете ли вы предложить решение этой задачи в общем случае, если мешков $n$?
\end{thm}

\begin{thm} \textbf{*}
Пятеро друзей сделали покупки на 5, 25, 125, 625 и 3125 франков. По выходе из магазина у
Жана остался 1 франк, у Поля -- 2, у Пьера -- 3, у Жака -- 4, а у Клода -- 5 франков. Если всю сумму, истраченную каждым, умножить на оставшуюся у него сумму и сложить эти пять произведений, то получится 9615. Сколько заплатил каждый из друзей?
\end{thm}

\begin{thm} \textbf{*}
Игрок ставит монету в 1 франк 7 раз подряд, затем вне зависимости от выигрыша или проигрыша, повышает ставку до 7 франков и снова играет 7 раз; затем 7 раз подряд он ставит 49 франков, затем 7 раз по 343 франка, затем 7 раз по 2401, потом 7 раз по 16807, и, наконец, 7 раз по 117649 франков. Сколько раз за всю игру он выиграл, если его выигрыш составил 777777 франков? (Найдите все решения)
\end{thm}

\begin{figure}
    \begin{minipage}{0.6\linewidth}
        \begin{thm}
            В клетках таблицы 7 $\times$ 7 записаны числа от 1 до 49. Выберем одно число и вычеркнем строку и столбец, в которых оно стоит. Затем выберем ещё одно число и т.д. Можете ли вы назвать сумму всех выбранных чисел 
        \end{thm}
    \end{minipage}
\hfill
    \begin{minipage}{0.35\linewidth}
        \begin{tabular}{ |c|c|c|c|c|c|c| } 
        \hline
        1 & 2 & 3 & 4 & 5 & 6 & 7 \\
        \hline
        8 & 9 & 10 & 11 & 12 & 13 & 14 \\
         & & & ... & & & \\
        43 & 44 & 45 & 46 & 47 & 48 & 49 \\
        \hline
        \end{tabular}
    \end{minipage}
\end{figure}

\noindent Наименьшее число, которое можно взять за основание системы счисления, -- это число два. Двоичная система счисления одна из самых старых. Она встречалась (правда в несовершенных формах) у некоторых племён Австралии и Полинезии. Некоторый недостаток этой системы состоит в том, что поскольку основание системы мало, то для записи даже не очень больших чисел приходится использовать много знаков. Однако удобство двоичной системы -- в необычайной простоте, её эквивалентность логическим функциям привело к широкому использованию двоичной системы в вычислительной технике.

\par \textbf{Задача.} Я загадал какое--то целое число от 1 до 1000. Как за 10 вопросов, на каждый из которых я буду отвечать «да» или «нет», отгадать это число?

\begin{prf}
    \textit{1--й вопрос:} разделите задуманное число на два с остатком. Делится оно на два без остатка? Если ответ «да», то запишем 0, если «нет», то 1.
    \par \textit{2--й вопрос:} разделите на два частное, полученное при предыдущем делении. Делится оно на два без остатка? Снова если ответ «да», то запишем 0, если «нет», то 1, и т.д. Каждый следующий вопрос будем составлять по тому же принципу: «Разделите на два частное, полученное при предыдущем делении. Делится ли оно без остатка?» Всякий раз пишем слева от уже написанного числа нуль при положительном ответе и единицу при отрицательном ответе. Повторив эту процедуру 10 раз, мы получим 10 цифр, каждая из которых есть нуль или единица. Легко видеть, что полученная запись является записью задуманного числа в двоичной системе. Действительно, система наших вопросов воспроизводит ту же процедуру, с помощью которой осуществляется перевод числа в двоичную систему счисления. При этом десяти вопросов достаточно, т.к. $2^{10}$ > 1000, и, следовательно, для записи числа от 0 до 1000 потребуется не более 10 знаков.

    Заметим, что решить задачу можно было и по-другому.
    \par
    \textit{1--й вопрос:} «Задуманное число больше 500?». Этот вопрос разбивает зону поиска на две половины: числа от 1 до 500 и числа от 501 до 1000. Ответ на вопрос сужает количество возможностей в два раза. Следующий вопрос зависит от ответа. Предположим, что был ответ «нет». Это значит, что следует рассматривать числа от 1 до 500. 
    \par \textit{2--й вопрос:} «Задуманное число больше 250?». (Если бы был ответ «да», мы бы задали вопрос: «Задуманное число больше 750?») Пусть был также дан ответ «нет». 
    \par \textit{3--й вопрос:} «Задуманное число больше 125?». Ответ: «нет». 
    \par \textit{4-й вопрос:} «Задуманное число больше 63?». Неважно, что полученные промежутки не равны. Важно, что каждый из промежутков не больше 64 = $2^6$. Отметим также, что и первый вопрос мог бы быть другим. Вместо 500 можно назвать любое число от 488 до 512. (\textbf{Контрольный вопрос:} почему?)
\end{prf}

\noindent Такой способ (когда область поиска последовательно делится пополам) активно используется, например, в программировании, и называется методом бинарного поиска.

\begin{thm}
    Оцените количество проб, достаточное для попадания точки в загаданный, заранее неизвестный, интервал длины 0,03см на отрезке длины 1м.
\end{thm}

\begin{thm}
    Ниже изображены пять табличек для следующего фокуса. Задумайте любое число от 1 до 31. Укажите, в каких табличках это число записано. После чего я угадаю ваше число. На чем основан этот фокус? По какому принципу записаны числа в таблицы?
\end{thm}

\begin{tabular}{ |c|c|c|c| } 
\hline
1 & 3 & 5 & 7 \\ 
\hline
9 & 11 & 13 & 15 \\ 
\hline
17 & 19 & 21 & 23 \\ 
\hline
25 & 27 & 29 & 31 \\ 
\hline
\end{tabular}
\hfill
\begin{tabular}{ |c|c|c|c| } 
\hline
2 & 3 & 6 & 7 \\ 
\hline
10 & 11 & 14 & 15 \\ 
\hline
18 & 19 & 22 & 23 \\ 
\hline
26 & 27 & 30 & 31 \\ 
\hline
\end{tabular}
\hfill
\begin{tabular}{ |c|c|c|c| } 
\hline
4 & 5 & 6 & 7 \\ 
\hline
12 & 13 & 14 & 15 \\ 
\hline
20 & 21 & 22 & 23 \\ 
\hline
28 & 29 & 30 & 31 \\ 
\hline
\end{tabular}
\hfill
\begin{tabular}{ |c|c|c|c| } 
\hline
8 & 9 & 10 & 11 \\ 
\hline
12 & 13 & 14 & 15 \\ 
\hline
24 & 25 & 26 & 27 \\ 
\hline
28 & 29 & 30 & 31 \\ 
\hline
\end{tabular}
\hfill
\begin{tabular}{ |c|c|c|c| } 
\hline
16 & 17 & 18 & 19 \\ 
\hline
20 & 21 & 22 & 23 \\ 
\hline
24 & 25 & 26 & 27 \\ 
\hline
28 & 29 & 30 & 31 \\ 
\hline
\end{tabular}