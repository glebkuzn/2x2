

\section{Подсчёт рёбер, двудольные графы, независимые множества.}

Как уже выводилось ранее, удвоенное количество ребер равно сумме всех степеней вершин графа. Часто идея подсчёта рёбер помогает решить задачу. Одна из идей состоит в подсчете концов рёбер (которая как раз и равна сумме степеней всех вершин). Лемма о рукопожатиях именно так и доказывается. Обобщением этой идеи является подсчёт концов рёбер двумя способами. Часто такая идея применяется, если рассматриваемый граф -- \textit{двудольный}.

\begin{dfn}
    Граф двудольный, если его множество вершин можно разбить на два подмножества так, что все рёбра будут соединять только элементы разных подмножеств.
\end{dfn}

\begin{thm}
    По окончании конкурса бальных танцев, в котором участвовали 7 мальчиков и 8 девочек, каждый назвал количество своих партнеров / партнёрш:3, 3, 3, 3, 3, 5, 6, 6, 6, 6, 6, 6, 6, 6, 6. Не ошибся ли кто--нибудь из них?
\end{thm}

\begin{prf}
    Рассмотрим граф, в котором вершины соответствуют танцорам, и две вершины соединены ребром, если они были партнёрами в танце. Ясно, что такой граф будет двудольным, в одной доле мальчики, в другой -- девочки. Количество партнёров / партнёрш, которое было названо -- это степень соответствующей вершины. Значит, эти степени можно разбить на два подмножества, степени доли мальчиков и степени доли девочек. Ясно, что число концов рёбер доли мальчиков в сумме равно числу концов рёбер доли девочек. Если степень 5 принадлежит одной доле, то к другой доле относятся степени 3 или 6, все они делятся на 3, поэтому сумма степеней одной доли делится на 3, а другой не делится. Следовательно, кто--то ошибся.  
\end{prf}

Часто подсчитывается не число рёбер, а число пар, троек и т.д. рёбер выходящих из одной вершины, как в следующем примере.

\begin{thm}
    Кружок по астрономии проводился в школе 20 раз. На каждое занятие приходило 5 человек. Известно, что никакие 2 школьника не встречались более чем на одном занятии. Доказать, что не менее 20 школьников посетили кружок.
\end{thm}

\begin{prf}
    Предположим, что это не так, и пусть школьников было не более 20. Рассмотрим двудольный граф, в верхней доле которого находятся вершины--школьники, а в нижней находятся вершины--занятия. Соединим их, если школьник посетил соответствующее занятие. Посчитаем число пар школьников двумя способами. С одной стороны их не более $\dfrac{20 \times 19}{2} = 190$. С другой стороны это число равно числу пар рёбер выходящих из вершины в нижней доле (занятия), т.е. их $20 \times (\dfrac{4 \times 5}{2}) = 200$. Противоречие.
\end{prf}

Иногда считается количество ребер, выходящих из какого--либо множества.

\begin{dfn}
    Назовем \textit{независимым} множеством вершин графа такое множество его вершин, что никакие две из них не соединены ребром.
\end{dfn}

\begin{dfn}
    Назовем \textit{доминирующим} множеством такое множество D вершин графа, что любая вершина соединена ребром с вершиной из D.
\end{dfn}

\begin{dfn}
    \textit{Максимальное} независимое множество вершин -- независимое множество вершин, которое становится зависимым при добавлении любой вершины.
\end{dfn}

С помощью подсчёта ребер можно доказать следующее утверждение.

\fbox{\begin{varwidth}{0.95\textwidth}
    \textbf{Теорема.} Если степени всех вершин в графе G не превосходят $l$, то число элементов в любом максимальном независимом множестве не меньше $\dfrac{V}{l + 1}$.
\end{varwidth}}

\textbf{Идеи доказательства.} Если I это максимальное независимое множество, то из I существует ребро
\\ к любой вершине из J -- дополнения к I. Значит $V = |I| + |J| \leq |I| + l |I|$.
\\  
Отсюда уже выводится требуемое неравенство.    