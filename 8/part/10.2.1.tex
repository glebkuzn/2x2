\head{Ноябрь}{Листок 10. Теория графов. Подсчет рёбер и лемма Холла.}

\begin{thm}
    В шахматном турнире в один круг участвуют 11 шахматистов. В настоящее время среди любых трёх из них хотя бы двое ещё не сыграли. Доказать, что сыграно не более 30 партий.
\end{thm}

\begin{thm}
    У каждого из жителей города $N$ знакомые составляют не менее 30\% населения города. На выборы каждый из жителей ходит, только если баллотируется хотя бы один его знакомый. Доказать, что можно так провести выборы из двух кандидатов, что на них придёт не менее половины жителей.
\end{thm}

\begin{thm}
    В стране 1993 города, и из каждого выходит не менее 93 дорог. Известно, что из любого города можно проехать по дорогам в любой другой. Доказать, что это можно сделать с не более чем 62 пересадками.
\end{thm}

\begin{thm}
    В некотором парламенте 1600 депутатов, которые образовали 16000 комитетов, в каждом из которых не менее 80 депутатов. Доказать, что существуют два комитета, имеющие не менее 4 общих членов.
\end{thm}

\begin{thm}
    В некоторые 16 клеток доски 88 поставили по ладье. Какое наименьшее количество пар бьющих друг друга ладей могло при этом оказаться?
\end{thm}

\begin{thm}
    \textbf{Лемма Холла:} пусть есть двудольный граф $G$ с двумя долями -- верхней $U$ и нижней $V$. Через $v(X)$, где $X$ -- подмножество верхней доли, будем обозначать множество элементов в нижней доле инцидентных с $M$. Пусть для любого подмножества $X$ верхней доли верно, что $v(X) \geq |X|$. Доказать, что в таком случае в графе существует полное паросочетание, т.е. для каждой вершины верхней доли можно выбрать вершину нижней доли, соединённую с ней так, что никакие две выбранные вершины не совпадают.
\end{thm}

\begin{thm}
    Пусть есть двудольный граф такой, что его верхняя доля есть объединение множеств $A_1, A_2 ... A_k$, а $l_1, l_2 ...$ и $l_k$ -- фиксированные натуральные числа. Доказать, что если для любых $B_1, K B_k$, являющихся подмножествами $A_1, A_2 ... A_k$, верно, что $v(B_1) + K + v(B_k) \geq l_1 B_1 + K + l_k B_k$, то можно выбрать непересекающихся представителей для верхней доли, по $l_i$ для каждого элемента $A_i$.
\end{thm}

\begin{thm}
\begin{enumerate}[itemsep=0.05em]
    \item Доказать, что если в графе степени всех вершин не меньше $n$ и не больше $m$, то можно удалить часть рёбер так, что в полученном графе степени всех вершин будут заключены между $n - k$ и $m - k$.
    \item Доказать, что если в графе степени всех вершин не меньше $n$ и не больше $2n$, то можно удалить часть рёбер так, что в полученном графе степени всех вершин будут заключены между $m$ и $2m$, где $m$ от 1 до $n$. 
    \item Доказать то же самое если степени между $n$ и $kn$ для произвольного натурального $k$.
\end{enumerate}
\end{thm}

\begin{thm}
    В лагерь приехали пионеры. Каждый из них имеет от 50 до 100 знакомых. Доказать, что всем им можно выдать пилотки 1331 цвета так, что среди знакомых каждого найдётся 20 человек с разными цветами пилотки.
\end{thm}