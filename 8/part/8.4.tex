\head{Сентябрь}{Листок 8. Логика. Уровень 1. Проверочная работа.}

\begin{thm}
    За столиком в кафе собрались трое друзей: Белокуров, Рыжов и Чернов. Брюнет сказал Белокурову: «Любопытно, что один из нас блондин, другой брюнет, а третий рыжий, но ни у кого цвет волос не совпадает с фамилией». Какой цвет волос у каждого из них?
\end{thm}

\begin{thm}
    Однажды в летнем лагере за круглым столом собрались пять ребят родом из Москвы, Астрахани, Новгорода, Перми, Костромы. Их звали: Юра, Толя, Лёша, Коля, Витя. Москвич сидел между Витей и жителем Костромы, астраханец -- между Юрой и Толей, а напротив его сидели пермяк и Лёша. Коля никогда до этого не был в Астрахани, Юра не был в Москве и Костроме, а костромич с Толей регулярно переписываются. Определите, в каком городе живёт каждый из ребят.
\end{thm}

\begin{thm}
    Чего больше: пятниц, кроме тех пятниц, которые не являются тринадцатыми числами, или тринадцатых чисел, кроме тех, которые не являются пятницами?
\end{thm}

\begin{thm}
    Четверо ребят -- Алёша, Боря, Ваня и Гриша -- соревновались в беге. На следующий день на вопрос, кто какое место занял, они ответили так:
\par Алёша: «Я не был ни первым, ни последним». \hfill Боря: «Я не был последним». \par Ваня: «Я был первым». \hfill Гриша: «Я был последним».
\\ Известно, что три из этих ответов правильные, а один неверный. Кто был первым?
\end{thm}