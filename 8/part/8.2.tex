\head{Сентябрь}{Листок 8. Логика. Уровень 1.}

% Уровень 1

\begin{thm}
    Школьники Иванов, Петров и Никитин сидели на скамейке. Их звали Ваня, Петя и Никита. Неожиданно Никитин сказал Пете: «А ты знаешь, что среди нас нет человека, у которого фамилия образована от его имени?» Назовите полные имена мальчиков. Под полным именем мальчика будем подразумевать его имя и фамилию.
\end{thm}

\begin{thm}
    Вася, Коля, Илья и Женя живут в разных городах: Москве, Киеве, Париже и Берлине. Известно, что если Женя живёт в Москве, то Илья не парижанин. Если же Илья – москвич, то Вася живёт не в Париже. Кроме того, известно, что Вася никогда не был в Москве и Киеве, а Коля – в Берлине и Париже. Илье не довелось побывать в Киеве и Берлине. А Женя мечтает съездить в Париж. В каких городах живёт каждый из мальчиков?
\end{thm}

\begin{thm}
    Костя, Женя и Миша имеют фамилии: Орлов, Соколов и Ястребов. Какую фамилию имеет каждый мальчик, если Женя, Миша и Соколов – члены математического кружка, а Миша и Ястребов занимаются музыкой?
\end{thm}

\begin{thm}
    3 мальчика, приехавшие в выездную школу из разных городов, рассказывают о себе:
    \par
    Петя: «Я живу в Москве, а Миша в Волгограде».
    \par
    Миша: «Я живу в Москве, а Петя в Архангельске».
    \par
    Коля: «Я живу в Москве, а Петя в Волгограде».
    \\
    Их учитель, удивлённый противоречиями в сказанном, попросил их объяснить, где правда, а где ложь. Тогда ребята признались, что в сказанном каждым из них одно утверждение – правда, а второе – ложно. В каком городе живёт каждый из мальчиков?
\end{thm}

\begin{thm} $^n$
    Ниф-Ниф, Наф-Наф и Нуф-Нуф пришли в казино в серебряной, золотой и платиновой цепях. Перстни у них были из тех же металлов. У Ниф-Нифа цепь и перстень были изготовлены из одинакового металла. У Наф-Нафа ни цепь, ни перстень не были серебряными. У Нуф-Нуфа цепь была из платины, а перстень не из платины. У кого из чего были изготовлены украшения?
\end{thm}

\begin{thm}
    Три друга – Пётр, Роман и Сергей – учатся на математическом, физическом и химическом факультетах. Если Пётр – математик, то Сергей не физик. Если Роман не физик, то Пётр - математик. Если Сергей не математик, то Роман – химик. Сможете ли Вы определить специальность каждого?
\end{thm}

\begin{thm}
    Дима, Катя, Миша, Света и Юра проводили лето на даче вместе с родителями. Все дети разного возраста: от 4-x до 8-ми лет. У каждого из них своя любимая еда (бананы, мороженое, пиццa, спагетти, шоколад). И каждый чего-нибудь боится (грозы, пауков, привидений, собак, темноты). Определите сколько лет каждому из них, у кого какая любимая еда, и кто чего боится.

    \begin{enumerate}
        \item Девочки старше остальных, ни одна из них не боится темноты, и обе не любят шоколад.
        \item Света обожает пиццу и не страшится пауков
        \item Пятилетний ребёнок больше всего боится привидений.
        \item Шестилетний ребёнок боится грозы и равнодушен к шоколаду и спагетти.
        \item Самый младший ребёнок любит есть бананы, а самый старший не боится собак.
        \item Ни Дима (ему не пять лет), ни Миша не боятся ни темноты, ни пауков, и оба не любят бананы.
    \end{enumerate}
    \textit{\textbf{Замечание.}} Считается, что возраст выражается целым числом лет без учёта месяцев.
\end{thm}

\begin{thm}
    Четыре подруги пришли на каток, каждая со своим братом. Они разбились на пары и начали кататься. Оказалось, что в каждой паре «кавалер» выше «дамы», и никто не катается со своей сестрой. Самым высоким в компании был Паша Воробьев, следующим по росту – Юра Егоров, потом – Люся Егорова, Серёжа Петров, Оля Петрова, Дима Крымов, Инна Крымова и Аня Воробьева. Кто с кем катался?
\end{thm}

\newpage

\begin{thm}
    В одном из институтов учатся 4 друга. Самый младший учится на 1 курсе, а старший – на 4. Известно, что Борис – персональный стипендиат, Василий летом должен ехать на практику в Омск, а Иванов – домой в Донбасс. Николай курсом старше Петра, Борис и Орлов – коренные ленинградцы, Крылов в прошлом учебном году окончил школу и поступил на факультет, где учится Карпов. Борис иногда пользуется прошлогодними конспектами Василия. Надо определить имя и фамилию каждого из друзей и курс, на котором он учится.
    \\
    \textit{\textbf{Замечание.}} Чтобы получить персональную стипендию нужно отучиться в институте не менее года.
\end{thm}

% Некоторые задачи вырезаны из-за того, что они уже есть в Оверлифе, в файле 8.0

% Уровень 1а

\begin{thm}
    Однажды в летнем лагере за круглым столом собрались пять ребят родом из Москвы, Астрахани, Новгорода, Перми, Костромы. Их звали: Юра, Толя, Лёша, Коля, Витя. Москвич сидел между Витей и жителем Костромы, астраханец – между Юрой и Толей, а напротив его сидели пермяк и Лёша. Коля никогда до этого не был в Астрахани, Юра не был в Москве и Костроме, а костромич с Толей регулярно переписываются. Определите, в каком городе живёт каждый из ребят.
\end{thm}

\begin{thm}
    За столиком в кафе собрались трое друзей: Белокуров, Рыжов и Чернов. Брюнет сказал Белокурову: «Любопытно, что один из нас блондин, другой брюнет, а третий рыжий, но ни у кого цвет волос не совпадает с фамилией». Какой цвет волос у каждого из них?
\end{thm}

% Файл "Уровень 2" состоит из второй половины Оверлифного файла 8.0. Как итог я добавил единственную не дублированную задачу из 2 Уровня сюда.

\begin{center}
    \textbf{«Перевёртыши»}
\end{center}
В выездных школах достаточно популярна игра в перевёртыши – когда известная фраза зашифровывается путём замены каждого из слов на условно противоположное. Например: По морю ёлка плыла (Во поле берёза стояла). Вам предлагается расшифровать приведённые ниже перевёртыши.
\begin{enumerate}
    \item Быстро сутки тонут близко. 
    \item Год помирай, год тупи.
    \item Некоторые жертвы могут догадаться, откуда прыгают ёжики.
    \item От одной лисы побежишь – всех разгонишь. 
    \item Десять на машине, включая кошку.
    \item Вы читали, вы читали, ваши ноженьки взбодрились.
    \item Птичка в небе полетала и сто баксов потеряла. Села птичка на забор – продавала всем топор.
    \item Один, один, один зелёный папоротник. 
    \item Три начала, три квадрата, и в конце гайка.
    \item Стоит Скиф около моря. Слышит Скиф – на море лебедь.
    \item Сиди тут – известно где, убери это – известно кого.
    \item Ежедневно знойным мигом ты в долину убегаешь.
    \item Стоит лягушка замирает, смеётся в остановке.
    \item Сдохли у ребёнка три занудных львёнка.
    \item Жара без мглы – поганый вечер.
    \item Подобрали зайку с крыши, прикрепили зайке крылья.
    \item У лисы какой-то чёрт отнял буханку хлеба.
\end{enumerate}