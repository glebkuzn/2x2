\head{Ноябрь}{Листок 10. Графы. Теория 1.}

\epigraph{\textit{«Бамбук -- дерево, ёжик -– дерево, ёжики с бамбуком –- лес...»}}

\textbf{\textit{Напоминание.}} Связный граф, не имеющий циклов, называется \textit{деревом}.

\fbox{\begin{varwidth}{0.95\textwidth}
    \textbf{\textit{Лемма.}} Граф является деревом тогда и только тогда, когда каждые две его вершины соединены ровно одним путём с различными ребрами.
\end{varwidth}}

\textit{\underline{Замечание}}: иногда дерево определяют именно как граф, в котором любые две вершины соединены ровно одним путём с различными ребрами (т.е. простым путём).

\begin{dfn}
    Вершина, из которой выходит только одно ребро, называется \textit{висячей}.
\end{dfn}

\fbox{\begin{varwidth}{0.95\textwidth}
    \textbf{\textit{Лемма о висячей вершине.}} В любом дереве найдётся висячая вершина.
\end{varwidth}}

\begin{dok}
    Выберем произвольную вершину А и рассмотрим какое--нибудь выходящее из этой вершины ребро. Пусть это ребро АВ. Если из вершины В не выходит других рёбер, то эта вершина –- искомая. В противном случае отметим ребро АВ и продолжим наш путь по любому неотмеченному ребру, выходящему из вершины B и так далее. Заметим, что в строящемся таким образом пути ни одна вершина не встречается дважды, в противном случае получился бы цикл, а дерево циклов не имеет. Поэтому при наличии неотмеченных рёбер мы будем каждый раз переходить в новую вершину, а их конечное число. Следовательно, в конце концов наш путь закончится. Но закончиться он может только в висячей вершине.
\end{dok}

\fbox{\begin{varwidth}{0.95\textwidth}
    \textbf{\textit{Утверждение.}} Число вершин дерева на единицу больше числа его рёбер.
\end{varwidth}}

\begin{thm}
    В стране Древляндии 2012 городов, некоторые из которых соединены дорогами. При этом любые два города соединяет ровно один маршрут. Сколько в этой стране дорог?
\end{thm}

\begin{prf}
    Поскольку любые два города соединяет ровно один маршрут (т.е. путь), то граф дорог этой страны – дерево. Поэтому рёбер (дорог) на единицу меньше, чем вершин. Следовательно вершин 2011.
\end{prf}

\begin{dfn}
    Граф О называется \textit{остовом} связного графа G, если О имеет те же вершины, что и G, получается из G удалением некоторых рёбер, и является деревом.
\end{dfn}

\fbox{\begin{varwidth}{0.95\textwidth}
    \textbf{\textit{Лемма об остове.}} Любой связный граф имеет остов.
\end{varwidth}}

\begin{dok}
    Если граф уже является деревом, то в качества остова можно выбрать его самого. Пусть данный граф не является деревом. Тогда докажем, что в любом таком графе мы можем удалить одно ребро без потери связности. Действительно, если этот граф не дерево, то в нём есть цикл. Удалим в цикле одно ребро. Очевидно, что граф останется связным. Поскольку если две вершины были соединены путём, не проходящим через удалённое ребро, то для них ничего не изменилось. Если же путь проходил через удалённое ребро, то заменим это ребро путём по оставшемуся куску цикла. Следовательно, путь восстановлен и граф остался связным. Таким образом, можно удалять по одному ребру до тех пор, пока граф не станет деревом. Но тогда мы получим искомый остов.
\end{dok}

\begin{figure}[H]
\begin{minipage}{0.65\linewidth}\setlength{\parindent}{1.5em}
    \begin{ques}
     Может ли граф иметь несколько остовов?
    \end{ques}

    \begin{thm}
    Рома и Саша стирают у фигуры, приведённой на рисунке по одной стороне квадратиков. Нельзя стирать вершины квадратиков и сторону так, чтобы фигура распалась на две несвязные части. Кто выигрывает в эту игру, если начинает Рома?
    \end{thm}
\end{minipage}
\hfill
\begin{minipage}{0.3\linewidth}

        \begin{tabular}{ | m{0em} | m{0em} | m{0em} | m{0em} | m{0em} | m{0em} | m{0em} | m{0em} | } 
    \hline
   &  &  &  &  &  &  &  \\ 
     \hline
   &  &  &  &  &  &  &  \\ 
    \hline
   &  &  &  &  &  &  &  \\ 
    \hline
   &  &  &  &  &  &  &  \\ 
    \hline
   &  &  &  &  &  &  &  \\ 
    \hline
\end{tabular}
    
\end{minipage}
\end{figure} 

\begin{prf}
    У данной фигуры $8 \times 6 + 9 \times 5 = 93$ стороны маленьких квадратиков (ребер графа) и $6 \times 9 = 54$ вершины. Не получится стереть сторону лишь тогда, когда граф станет деревом. Но тогда при 54 вершинах должно будет остаться 53 ребра. Поэтому при любых ходах стереть можно ровно 40 отрезков и выиграет Саша, поскольку именно он делает чётные ходы. \textbf{\textit{Ответ}}: выиграет Саша
\end{prf}