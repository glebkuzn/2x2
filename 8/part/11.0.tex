\head{Март}{Листок 11. Делимость целых чисел -- 2.}

\epigraph{\textit{Прошу -- забудь всё, чему ты учился в школе, потому что ты этому не научился.}}{\textit{Э.Ландау «Основы анализа»}}

\section{Часть 1.}

Если $a$ и $b \in \mathbb{Z}$, причём $b > 0$, то существует такое \textit{q (частное)} $\in \mathbb{Z}$, что $a = bq + r$, где \textit{r (остаток)} $\in \mathbb{Z}$ и $0 \leq r < b$. Числа $q$ и $r$ определяются по данным $a$ и $b$ единственным образом: если $r = 0$, мы получаем случай, когда $a$ делится на $b$. В этом случае $b$ называют \textit{делителем} $a$.
\\ Пусть $a$ и $b$ -- два целых числа, не равные одновременно нулю. Рассмотрим все числа, на которые делятся и $a$ и $b$ одновременно, т.е. рассмотрим все \textit{общие делители} $a$ и $b$. Выберем из них наибольший и назовем его \textit{наибольшим общим делителем}. Дальше будем обозначать наибольший общий делитель чисел $a$ и $b$ через НОД($a,~b$) или, для краткости, ($a,~b$). \textit{Наименьшим общим кратным} чисел $a$ и $b$ (НОК($a,~b$) или, для краткости, $[a, b]$) называется наименьшее натуральное число, которое делится на $a$ и $b$. (Вообще, общее кратное двух
или более чисел это число, делящееся на все эти числа)
\par
Пусть $a_1, a_2, ... , a_n$ -- целые числа, не равные одновременно нулю. Будем обозначать:
\begin{center}
    \textbf{НОД($a_1, a_2, ... , a_n$)} или $(a_1, a_2, ... , a_n)$, и \textbf{НОК($a_1, a_2, ... , a_n$)} или $[a_1, a_2, ... , a_n]$.
\end{center}

\begin{dfn}
    Если НОД($a,~b$) = 1, то числа $a$ и $b$ называются взаимно простыми.
\end{dfn}

\begin{ex} \label {10.7 ex1}
    Выберите из следующих чисел все возможные пары взаимно простых чисел: 14, 18, 21, 35, 45, 60, 78, 99.
\end{ex}

\begin{thm}
    При каких натуральных $n$ будут взаимно простыми числа: $7n + 1$ и $5n + 2$?
\end{thm}

\begin{prf}
    Заметим, что первое число не делится на 7, а второе на 5 при любых значениях $n$. Поэтому числа $7n + 1$ и $5n + 2$ взаимно просты тогда и только тогда, когда взаимно просты числа $5(7n + 1) = 35n + 5$ и $7(5n + 2) = 35n +14$. Но если какие--то два числа имеют общий делитель, то их разность имеет тот же самый делитель. Разность чисел $35n + 5$ и $35n + 14$ равна 9. Поэтому, если и есть общий делитель, то это либо 3, либо 9.
    \par Вернёмся к нашим числам. Если $n$ кратно 3 или при делении на 3 даёт остаток 1, то ни одно из чисел на 3 не делится. Если $n$ при делении на 3 даёт остаток 2, то оба числа делятся на 3 и поэтому не взаимно просты.
    \par \textbf{Ответ:} при $n = 3k$ или $n = 3k + 1$, где $k$ -- любое натуральное.
\end{prf}

\begin{thm}
    При каких натуральных $n$ число $n^2 + 1$ делится на $n + 3$?
\end{thm}

\begin{prf}
    \textit{1 способ.} Разделим $n^2 + 1$ на $n + 3$ с остатком: $n^2 + 1 = (n + 3)(n – 3) + 10$. Следовательно, если $n^2 + 1$ и $n + 3$ имеют общий делитель, то 10 делится на этот делитель. Поэтому общим делителем может быть только 2, 5 или 10. Достаточно разобрать случаи 2 и 5. Очевидно, что на 2 оба числа делятся, если $n$ нечётно. Поэтому $n$ должно быть чётно. Разберём теперь случай 5 как общего делителя. Для этого рассмотрим $n$ с точки зрения его делимости на 5. Если $n$ кратно 5, то оба числа на 5 не делятся. Осталось рассмотреть остатки 1, 2, 3 и 4.
    \par
    \textit{2 способ.} Разделим $n^2 + 1$ на $n + 3$ с остатком: $n^2 + 1 = (n + 3)(n - 3) + 10$. Или, другими словами, $\dfrac{n^2 + 1}{n + 3} = (n - 3) + \dfrac{10}{n + 3}$. Если $n^2 + 1$ делится на $n + 3$ -- целое число, т.е. 10 делится на $n + 3$. А это возможно только когда $n + 3$ равно 5 или 10. Следовательно, $n$ равно 2 или 7.
    \par \textbf{Ответ:} при $n = 2$ или $n = 7$.
\end{prf}

\begin{thm} \label{10.8 thm3}
    Докажите, что любое общее кратное чисел $a$ и $b$ делится на НОК($a,~b$). (Другими словами: если какое--то число делится на $a$ и $b$, то данное число делится и на их наименьшее общее кратное.)
\end{thm}

\newpage

\begin{prf}
    Пусть $K$ –- общее кратное чисел $а$ и $b$, которое не делится на НОК($a,~b$) = $m$. Тогда можно поделить $K$ на $m$ с остатком: $K = mt + r$, где $0 < r < m$. Следовательно, $r = K - mt$. Т.к. $K$ делится на $a$ и $m$ делится на $а$, то $r$ делится на $a$. Аналогично, $r$ делится на $b$. Поэтому $r$ также является общим кратным чисел $а$ и $b$. Но $r < m =$ НОК($a,~b$), что противоречит определению наименьшего общего кратного. Мы получили противоречие из предположения существования общего кратного чисел $a$ и $b$, не делящегося на НОК($a,~b$). Тем самым утверждение доказано.\footnotemark
\end{prf}\footnotetext{Этот метод доказательства -- разновидность «правила крайнего». Его смысл состоит в том, что предполагается, что некое утверждение верно (в данном случае утверждение о том, что существует кратное, не делящееся на НОК) и рассматривается «крайний» элемент, в данном случае НОК, после чего в результате рассуждений получается противоречие с тем, что наш выбранный «крайний» элемент действительно является «крайним» (в данном случае наименьшим кратным). Отсюда делается вывод о несправедливости исходного предположения.}

\begin{center}
  \fbox{\begin{varwidth}{0.95\textwidth}
    \begin{thrm} \label{11.0 thrm1}
        Для любых натуральных чисел $a$ и $b$ справедливо $ab = НОД(a, b) \times НОК(a, b)$.
    \end{thrm}
    \par
    \textbf{\textit{Важное следствие:}} если $a$ и $b$ взаимно просты, то $НОК(a, b) = ab$.
    \begin{thrm} \label{10.7 thrm1}
        Пусть $a$ и $b$ – целые числа, $ab \neq 0$, $d = НОД(a, b)$. Тогда существуют целые числа $u$ и $v$, такие, что $au + bv = d$.
    \end{thrm}
    \par
    \textbf{\textit{Пример.}} $НОД(18, 30) = 6$, $6 = 18 \times 2 + 30 \times (-1)$, т.е. $u = 2, v = -1$.
    \par
    \textbf{\textit{Важное следствие:}} если числа $a$ и $b$ взаимно просты, то существуют целые числа $u$ и $v$, такие, что $au + bv = 1$.
    \end{varwidth}}  
\end{center}

Доказательство теоремы \ref{10.7 thrm1}. 

\begin{enumerate}[itemsep=0.05em]
    \item Пусть $M$ -- множество всех натуральных чисел, которые можно представить в виде $ax + by$ с целыми $x$ и $y$. Заметим, что сами числа $a$ и $b$ (или противоположные к ним, если они отрицательны) также можно представить в таком виде; например, если $a$ > 0, то $a$ = $a \times 1 + b \times 0 \in M$. Если же $a < 0$, то $-a = a \times (-1) + b \times 0 \in M$; аналогично для $b$. Ещё пример: если $a + b > 0$, то $a + b=a \times 1 + b \times 1 \in M$. Таким образом, мы показали, что такие числа существуют.\footnote{Заметим, что было вовсе не обязательно предъявлять два примера для доказательства существования чисел, представимых в таком виде. Достаточно было одного примера.}
    \item Пусть $d$ -- наименьшее из чисел, принадлежащих множеству $M$. Тогда, поскольку $d$ натуральное число, $d = au + bv > 0$, где $u$ и $v$ -- целые числа.
    \item Докажем, что $d | a$. Допустим противное: $d \dag a$. Разделим число $a$ на $d$ с остатком: $a = dq + r$, $0 < r < d$. Но тогда $r = a - dq = a - q(au + bv) = a(1 - qu) + b(-qv) = au_1+ bv_1$.
    \\ Следовательно, натуральное число $r$, меньшее $d$, тоже можно представить в виде $r = au_1 + bv_1$, где $u_1 = 1 - qu$, $v_1 = -qv$. Но это противоречит тому, что $d$ -- наименьшее число, которое можно в таком виде представить, т.е. предположение, что $d \dag a$ неверно, а значит $d | a$. Аналогично доказывается, что $d | b$.
    \item Таким образом, $d$ -- общий делитель чисел $a$ и $b$. Докажем, что $d = НОД(a, b)$, то есть, если найдётся другое натуральное число $d_1$, такое, что оно делит $a$ и делит $b$, то $d_1 \leq d$.
\end{enumerate}
Теорема доказана. \hfill \qedsymbol

\textit{\underline{Замечание.}} Отметим, что мы воспользовались тем же методом, что и при доказательстве утверждения задачи \ref{10.8 thm3}.

\textbf{Вопрос.} Зачем нужно было доказывать, что найдутся числа, представимые в рассматриваемом виде? В каком месте доказательства это было использовано?

\begin{ex} \label{10.7 ex2}
    Пусть $au + bv = 2$. Верно ли, что НОД($a,~b$) = 2?
\end{ex}

\begin{ex} \label{10.7 ex3}
    Найдите НОД($a$, $a + 1$).
\end{ex}

Ответы к упражнениям:

\par
\ref{10.7 ex1}. 6 пар (14, 45); (14, 99); (18, 35); (35, 45); (35, 78); (35, 99)
\par
\ref{10.7 ex2}. нет, например $2 \times 5 - 4 \times 2 = 2$, но НОД(5, 2) = 1. 
\par
\ref{10.7 ex3}. 1.