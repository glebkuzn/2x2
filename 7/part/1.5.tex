\head{Сентябрь}{Листок 1. Четность. Уровень 2.}
\epigraph{\textit{Беды обычно приходят парами — пара за парой, пара за парой, пара за парой...}}{Следствие Кона из закона Мерфи}

\begin{thm}\label{1.10}
	Полный комплект костей домино выложен в цепочку. На одном конце оказалась пятерка. А что могло оказаться на другом?
\end{thm}

%\begin{prf}
%	Рассмотрим множество половинок всех доминошек. Всего доминошек 28, при этом каждое число (от 0 до 6) присутствует ровно на 8 половинках. Заметим, что все половинки кроме двух крайних разбиты на пары с одинаковыми цифрами. Это означает, что среди них любое число (от 0 до 6) встречается четное число раз. Тогда, если не рассматривать неизветсный конец, пятерка встречается на одном конце и еще на четном количестве мест, а все остальные числа встречаются на четном количестве мест. Отсюда однозначно следует, что на втором конце тоже пятерка.
%\end{prf}

\begin{thm}\label{1.11}
	Из полного набора домино, подаренного родителями, Ваня потерял все кости с «пустышками». Сможет ли теперь кто-нибудь выложить оставшиеся кости в ряд?
\end{thm}

%\begin{prf}
%	Не сможет. Предположим обратное: пусть кости разложены в ряд. Все половинки, кроме двух крайних, объединяются в пары с равными числами. На двух крайних числа могут быть как равные, так и нет (мы не можем ссылаться в этом месте на предыдущую задачу, так как часть косточек потеряна). Следовательно, все числа, за исключением двух, заведомо появляются четное число раз. Но после потери всех пустышек осталось ровно по 7 экземпляров каждой из 6 цифр 1,2,3,4,5,6. 
%\end{prf}

\begin{thm}\label{1.12}
	На бирже в городе Нью-Васюки ежедневно в 10.00 проходят торги. Рано утром 1 января N-го года цены на акции фирм «Вася Inc.» и «Петя и Ко» были один и два рубля соответственно. Вечером 31 декабря того же года цены стали снова теми же. Лёша установил, что цены на акции этих фирм всегда были различны, каждый день изменялись и все время были либо один, либо два рубля. Докажите, что прошедший год был високосным.
\end{thm}

\begin{thm}$^n$\label{1.21} В разные моменты времени из пунктов А и В выехали навстречу друг другу велосипедист и мотоциклист. Встретившись в точке С, они тотчас развернулись и поехали обратно. Доехав до своих пунктов, они опять развернулись и поехали навстречу друг другу. На этот раз они встретились в точке D и, развернувшись, вновь поехали к своим пунктам. И т.д. В какой точке отрезка АВ произойдет их 2019 встреча?
\end{thm}

\begin{thm}\label{1.14}Можно ли выписать в ряд по одному разу цифры от 1 до 9 так, чтобы между единицей и двойкой, двойкой и тройкой, \dots, восьмеркой и девяткой было нечетное число цифр?
\end{thm}

\begin{thm}\label{1.15}7М класс упражняется в счете. Анатолий Анатольевич написал на доске число 2011. После чего каждый ученик вышел к доске, прибавил или вычел 17 или 13 и записал получившийся результат. Когда каждый из 20 учеников вышел по одному разу, на доске оказалось написано число 2012. Анатолий Анатольевич посмотрел на доску и расстроился. Докажите, что кто-то из учеников ошибся.
\end{thm}

\begin{thm}\label{1.16}
	У Вини-Пуха было 2019 горшочков меда. Кристофер Робин принес или забрал 9 горшочков, что именно - Пух не помнит. На следующий день Кристофер Робин снова пришел и принес или забрал 8 горшочков, на следующий день - 7 и так далее. Наконец Кристофер Робин пришел и принес или забрал один горшочек.  а) Могло ли у Винни-Пуха на 10 день оказаться горшочков столько же, сколько и было в самом начале, то есть 2019? б) Сколько вообще горшочков меда могло быть у Вини-Пуха на 10 день, если все это время он мед не ел?
\end{thm}
\begin{center}
	\textbf{Разбиение на пары}
\end{center}
\begin{thm}
	Докажите, что число способов расставить на доске 8 ферзей так, чтобы они не били друг друга, четно.
\end{thm}

\begin{thm}
	Лиза сложила лист бумаги пополам, после чего вырезала из него фигурку. После разворачивания фигурка оказалась шестиугольником. Сколько различных значений могут принимать длины его сторон?
\end{thm}

\begin{thm}
	Пусть билеты для проезда в наземном транспорте\footnote{Мы не будем здесь рассматривать все возможные билеты для проезда. Безусловно, сейчас существуют и семизначные, и восьми- и даже тринадцатизначные номера для проездных документов.}  имеют номера от 000000 до 999999. Назовем билет «счастливым», если сумма первых трех цифр равна сумме трех последних его цифр. Докажите, что число таких «счастливых» билетов четно. (Для решения задачи вовсе не обязательно считать точное количество «счастливых» билетов)
\end{thm}

\begin{thm}
	7Ю класс уселся за круглый стол. Елена Юрьевна между соседями-мальчиками положила по ручке, между соседями-девочками - по карандашу, а если рядом сидели мальчик и девочка, то между ними она положила по тетрадке.\\
	а) Докажите, что ей понадобиться четное число тетрадок.\\
	б) А может ли она обойтись нечетным количеством карандашей?\\
	в) Каким минимальным числом ручек она могла обойтись, если в классе 17 мальчиков и 5 девочек?
\end{thm}

\begin{thm}\label{an1.3}
	а) В ряд выписаны числа от 1 до 10. Можно ли расставить между ними знаки «+» и «-» так, чтобы значение полученного выражения было равно нулю?\\ б) А если выписаны числа от 1 до 2019, можно ли получить 17?
\end{thm}

\begin{thm}$^\ast$
	Для уроков информатики Михаил Владимирович приготовил 7 карточек, на которых были написаны числа от 5 до 11. Он их перемешал и предложил Глебу и Юле. Глеб взял себе три карточки, Юля - две, а оставшиеся две Михаил Владимирович отдал Ване, который их тут же потерял. Глеб сразу сказал Юле: «Я точно знаю, что сумма чисел на твоих карточках четна», и оказался абсолютно прав. Какие числа были написаны на карточках у Глеба?
\end{thm}

\begin{thm}$^\ast$ \label{1.30} На 99 карточках пишут числа 1, 2, ..., 99, перемешивают их, раскладывают чистыми сторонами вверх и снова пишут числа 1, 2, ..., 99. Для каждой карточки складывают два её числа и 99 полученных сумм перемножают. Докажите, что результат чётен.
\end{thm}
\textit{ Напоминаем, что задачи, отмеченные значком ${}^{ n}$, приниматься в устном форме не будут. В данном случае это единственная задача \ref{1.21}. Решение таких задач надлежит выполнить в письменном виде дома и принести на занятие спецматематики. Через некоторое время часть задач будет разобрана, после чего решения этих задач приниматься не будет. Задачи, отмеченные звездочками * не являются обязательными для получения следующего задания. Их решение может приниматься после перехода на следующий уровень.}     
\begin{flushright}
	\textit{Желаем успеха!}
\end{flushright}
\newpage