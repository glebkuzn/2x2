\section{Как проходит занятие}
Поскольку это первое занятие такого типа, то сначала школьникам объясняются «правила игры», отвечая на все возникающие вопросы. Далее мы предлагаем не выдавать сразу задачи для решения, а сначала поговорить в целом, как школьники понимают четные и нечетные числа, их свойства.

На доске выписываем равенства :

$ Ч+Ч=Ч $\hfill
$ Ч+Н=Н $\hfill
$ Н+Ч=Н $\hfill
$ Н+Н=Ч $

Первый факт можно предложить доказать, используя только определение чётного числа -- «Число называется чётным, если его можно разделить на две равные части».\footnote{Предварительно мы оговариваем, что пока имеем дело только с целыми числами и «делимость на две равные части» мы подразумеваем делимость целого числа на целочисленные части.}

\begin{prf}
	Пусть число А четное и число В четное. Докажем, что число $ А+В $ также четное. По определению это значит, что число  А делится на 2 и число В делится на 2.
	То есть $ А = a+a $ и $ В=b+b $. Тогда $ А+В = (a+b) + (a+b) $
	
	Тут можно проиллюстрировать свою речь картинками типа:
	
	$\hfill  А=\blacktriangle+\blacktriangle \hfill В =\blacktriangledown + \blacktriangledown \hfill  А+В = \blacklozenge+\blacklozenge \hfill$ 
	
	То есть сумму можно представить в виде суммы двух равных целых слагаемых. Что и требовалось доказать. 
\end{prf}

Аналогично для разности четных чисел. Можно предложить доказать этот факт самостоятельно или вынести позже в проверочную работу.

Еще один факт, который стоит доказать. А именно то, что сумма четного и нечетного числа нечетна. 

\begin{prf}
	Пусть $ А $ -- четное, $ В $ -- нечетное. Докажем, что $ А+В $ -- четное.
	Будем доказывать «методом от противного». Предположим, что требуемое неверно и А+В -- четное. Тогда $ В = (А+В) - А$ -- разность двух четных чисел и должно быть четным. Получили противоречие. То есть $ А+В $ -- нечетно. 
\end{prf}

Далее можно всем вместе разобрать часть задач и упражнений из теоретического листка (он приведен ниже). Рекомендуется сначала обсудить задачи 1-3 этого листка и только после разбора выдать этот листок школьникам. 

После выдачи листка можно предложить школьникам проверить себя -- самостоятельно решить упражнения и поверить себя по ответам в конце листка. Мы обычно обсуждаем все вместе задачи на доказательство, так как в этом возрасте такие задания идут наиболее тяжело. После этого, если вопросов нет, выдать первый листок уже с заданиями.

Обращаем внимание, что первый листок включает в себя не только сами задачи, но и некоторые комментарии к ним и дополнительно «правила игры»